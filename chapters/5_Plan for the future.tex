\chapter{KẾ HOẠCH TƯƠNG LAI}

Ở báo cáo này, nhóm đã đạt được một số thành quả như sau:
\begin{itemize}
	\item Hiểu rõ hơn về bài toán định ví hóa trực quan; bao gồm định nghĩa bài toán và các áp dụng của bài toán vào thực tế.
	\item Thông qua việc khảo sát các bài báo khoa học, nhóm đã hiểu được nhiều cách tiếp cận với bài toán định vị hóa trực quan cũng như các ưu điểm và khuyết điểm của từng phương pháp.
	\item Học cách sử dụng nền tảng Colab để có thể chạy các mô hình nhằm thí nghiệm và kiểm thử kết quả.
	\item Ngoài ra, nhóm đề xuất được một phương pháp được cho là tối ưu nhất làm giải pháp cho bài toán định vị trực quan.
\end{itemize} 

Tuy nhiên nhóm vẫn còn tồn tại một số vấn đề và hạn chế:
\begin{itemize}
	\item Chưa thể huấn luyện lại hoàn toàn mô hình do hạn chế về mặt thiết bị phần cứng.
	\item Chưa thể kết nối các mô hình trong phương pháp đề xuất, hoạt động của các mô hình vẫn còn mang tính riêng biệt.
\end{itemize}

Để khắc phục những thiếu sót cũng như cải thiện và hoàn thành đề tài của mình, nhóm dự định vào giai đoạn tiếp theo sẽ:
\begin{itemize}
	\item Xây dựng một bộ khung để có thể ghép và chạy cả hai mô hình trong cùng một lúc.
		\begin{itemize}
			\item Dự kiến khoảng thời gian thực hiện là một tháng.
			\item Kết quả mong muốn ở bước này là một bộ khung để chạy cả quá trình định vị hóa trực quan trong một thao tác: chỉ cần nhận ảnh đơn 2D vào, mô hình sẽ trả về kết quả là vị trí 6 DoF mà không cần đến các thao tác nào khác của người dùng trong lúc chạy.
		\end{itemize}
	\item Đồng thời, nhóm cũng đề xuất một quy trình để kết hợp cả hai mô hình thành một mô hình đầu-cuối.
		\begin{itemize}
			\item Dự kiến khoảng thời gian thực hiện là hai tháng.
			\item Kết quả mong muốn ở bước này là một mô hình học sâu để chạy cả quá trình định vị hóa trực quan: đầu vào là ảnh đơn 2D và đầu ra của mô hình sẽ là vị trí 6 DoF của máy ảnh chụp.
		\end{itemize}
\end{itemize}
