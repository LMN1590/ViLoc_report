\chapter{TỔNG KẾT}

Qua quá trình tìm hiểu, nghiên cứu, thử nghiệm và đánh giá, chúng tôi đã gặt hái được các thành quả sau: chúng tôi hiểu rõ hơn về bài toán định ví hóa trực quan, bao gồm định nghĩa bài toán và các ứng dụng của bài toán trong thực tế. Thông qua việc khảo sát các bài báo khoa học, chúng tôi đã hiểu được nhiều cách tiếp cận đa dạng với bài toán định vị hóa trực quan cũng như các ưu điểm và khuyết điểm của từng phương pháp.

Sau khi đánh giá các hướng tiếp cận trong quá trình khảo sát, chúng tôi đề xuất một phương pháp sử dụng mô hình hai bước, kết hợp giữa tác vụ nhận diện địa điểm trực quan và tác vụ hồi quy tư thế máy ảnh tương đối, để giải quyết bài toán định vị trực quan. Chúng tôi đề xuất nhiều hướng phát triển khác nhau nhằm tìm cách khai phá tiềm năng của hướng tiếp cận này: chiến lược khai phá mới làm hướng cải thiện và tinh chỉnh mô hình VPR, tái sắp xếp kết quả, áp dụng trung bình trọng số và loại trừ kết quả sai sót với ngưỡng tương quan. Về mặt kết quả, các thí nghiệm đo đạc và đánh giá đã cho thấy mô hình đề xuất của chúng tôi luôn đạt sai số vị trí nhỏ nhất với các tập dữ liệu được sử dụng.

Tuy nhiên chúng tôi gặp phải một số hạn chế: kết quả kiểm thử cho thấy chiến lược khai phá mới có thể chưa hoàn toàn phù hợp với mô hình, thể hiện thông qua độ cải thiện giữa mô hình tinh chỉnh không quá nhiều so với mô hình cơ sở. Các đề xuất phát triển trên lý thuyết phần nhiều chưa thể hiện ảnh hưởng tích cực đến kết quả. Việc kiểm thử tác vụ VPR trên tập dữ liệu Pittsburgh250k-test cho thấy dù mô hình pipeline đã cải thiện sai số về vị trí, mô hình của chúng tôi vẫn gặp khó khăn về mặt tính toán góc quay.

Để khắc phục những thiếu sót cũng như phát triển và hoàn thành đề tài của mình, chúng tôi dự định vào giai đoạn tiếp theo sẽ tiếp tục tìm hiểu các chiến lược khai phá mới với tiềm năng phù hợp hơn với mô hình. Ngoài ra, chúng tôi sẽ tiến hành nghiên cứu và tìm hiểu các hướng phát triển khác nhằm giúp đẩy mạnh hiệu quả toàn thể mô hình đề xuất trong tương lai.
