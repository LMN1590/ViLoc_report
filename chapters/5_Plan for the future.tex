\chapter{TỔNG KẾT}

Tổng kết lại, ở báo cáo này, nhóm chúng tôi đã đạt được một số thành quả sau: chúng tôi hiểu rõ hơn về bài toán định ví hóa trực quan, bao gồm định nghĩa bài toán và các áp dụng của bài toán vào thực tế. Thông qua việc khảo sát các bài báo khoa học, chúng tôi đã hiểu được nhiều cách tiếp cận với bài toán định vị hóa trực quan cũng như các ưu điểm và khuyết điểm của từng phương pháp. 

Sau khi đánh giá các hướng tiếp cận trong quá trình khảo sát, chúng tôi đề xuất một phương pháp sử dụng mô hình hai bước, kết hợp giữa tác vụ nhận diện địa điểm trực quan và tác vụ hồi quy tư thế máy ảnh tương đối, để giải quyết bài toán định vị trực quan. Với mô hình kết hợp này, chúng tôi đề xuất nhiều hướng phát triển khác nhau nhằm tìm cách khai phá tiềm năng của hướng tiếp cận này: đề xuất chiến lược khai phá mới làm hướng cải thiện và tinh chỉnh mô hình VPR, reranking kết quả, áp dụng bình quân trọng số và loại trừ kết quả sai sót với ngưỡng inlier. Về mặt kết quả, các thí nghiệm đo đạc và đánh giá đã cho thấy mô hình đề xuất của chúng tôi luôn đạt sai số vị trí nhỏ nhất với các tập dữ liệu được sử dụng.

Tuy nhiên chúng tôi gặp phải một số vấn đề và hạn chế: kết quả kiểm thử cho thấy chiến lược khai phá mới có thể chưa hoàn toàn phù hợp với mô hình, thể hiện qua việc độ cải thiện giữa mô hình tinh chỉnh không quá nhiều so với mô hình cơ sở. Các đề xuất phát triển của chúng tôi phần nhiều chưa có quá nhiều cải thiện cho kết quả. Kết quả kiểm thử tác vụ VPR trên tập dữ liệu Pittsburgh250k-test cho thấy dù đã cải thiện về vị trí, mô hình của chúng tôi vẫn gặp khó khăn về mặt tính toán góc quay.

Để khắc phục những thiếu sót cũng như cải thiện và hoàn thành đề tài của mình, chúng tôi dự định vào giai đoạn tiếp theo sẽ tiếp tục tìm hiểu, cải thiện chiến lược khai phá đã được đề xuất trong bài báo cáo để phù hợp hơn với mô hình. Ngoài ra, chúng tôi sẽ tiến hành nghiên cứu và tìm hiểu các hướng phát triển khác nhằm giúp mô hình đề xuất đạt hiệu quả tốt hơn.
