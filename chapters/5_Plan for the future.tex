\chapter{TỔNG KẾT}

Ở báo cáo này, chúng tôi đã đạt được một số thành quả như sau:
\begin{itemize}
	\item Hiểu rõ hơn về bài toán định ví hóa trực quan; bao gồm định nghĩa bài toán và các áp dụng của bài toán vào thực tế.
	\item Thông qua việc khảo sát các bài báo khoa học, chúng tôi đã hiểu được nhiều cách tiếp cận với bài toán định vị hóa trực quan cũng như các ưu điểm và khuyết điểm của từng phương pháp.
	\item Chúng tôi đề xuất được một phương pháp sử dụng mô hình hai bước, kết hợp giữa tác vụ nhận dạng địa điểm trực quan và tác vụ hồi quy tư thế máy ảnh, để giải quyết bài toán định vị trực quan.
	\item Chúng tôi đề xuất miner mới làm hướng để cố gắng cải thiện và tinh chỉnh mô hình MixVPR.
	\item Kết quả kiểm thử cho thấy mô hình hai bước của chúng tôi hoạt động tốt hơn hẳn so với các mô hình được so sánh khác trên tác vụ RPR.
\end{itemize}

Tuy nhiên chúng tối vẫn gặp phải một số vấn đề và hạn chế:
\begin{itemize}
	\item Chưa thể huấn luyện lại hoàn toàn mô hình VPR do hạn chế về mặt thiết bị phần cứng.
	\item Kết quả kiểm thử cho thấy miner chưa hoàn toàn phù hợp với mô hình, thể hiện qua việc độ cải thiện giữa mô hình tinh chỉnh không quá nhiều so với mô hình cơ sở.
	\item Kết quả kiểm thử tác vụ VPR trên tập dữ liệu Pittsburgh250k-test cho thấy dù đã cải thiện về vị trí, mô hình của chúng tôi vẫn gặp khó khăn về mặt tính toán góc quay.
\end{itemize}

Để khắc phục những thiếu sót cũng như cải thiện và hoàn thành đề tài của mình, chúng tôi dự định vào giai đoạn tiếp theo sẽ:
\begin{itemize}
	\item Tiếp tục tìm hiểu, cải thiện miner được đề xuất trong bài báo cáo để phù hợp hơn với mô hình.
	      \begin{itemize}
		      \item Dự kiến khoảng thời gian thực hiện là một đến hai tháng.
		      \item Kết quả mong muốn: độ chính xác của mô hình sau khi tinh chỉnh vượt trội so với mô hình cơ sở gốc.
	      \end{itemize}
\end{itemize}
