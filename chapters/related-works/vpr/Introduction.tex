Trước đây, việc định vị trực quan trên quy mô lớn (large-scale visual localization) được coi là một vấn đề truy xuất hình ảnh\cite{2022arXiv220105816X}. Vị trí cho hình ảnh truy vấn được xác định bởi hình ảnh tương tự nhất được lấy từ cơ sở dữ liệu. Tuy nhiên, để đáp ứng nhu cầu xác định vị trí của ảnh với độ chính xác trên 6 bậc tự do (6 Degrees of Freedom), việc sử dụng mô hình 3D để ước tính tư thế của máy ảnh ngày càng được các nhà nghiên cứu đề xuất và sử dụng. Sử dụng phương pháp này, bài toán định vị trực quan có thể được chia thành hai bước: truy xuất hình ảnh và định vị tư thế máy ảnh từ hình ảnh truy xuất được.