\subsection{Phân tích và tổng hợp}

Trước sự phát triển của mạng học sâu tích chập trong lĩnh vực thị giác máy tính và hiệu năng của các đặc trưng toàn cục huấn luyện được, có thể nhận định rằng, trong tương lai, bài toán nhận dạng địa điểm trực quan sẽ tiếp tục được phát triển với học biểu diễn đặc trưng toàn cục là chính và sẽ kết hợp với quá trình khai phá mối quan hệ cục bộ giữa các đặc trưng. Hướng phát triển này cho phép các nhà nghiên cứu huấn luyện và chạy mô hình với tác động tài nguyên ít mà vẫn giữ được độ bền của mô hình trước các yếu tố thay đổi đến từ ảnh địa danh. Sau khi trích xuất đặc trưng toàn cục, mô hình VPR thường sử dụng quá trình tổng hợp đặc trưng (aggregation) hoặc gộp đặc trưng (pooling) để khai phá mối quan hệ cục bộ. Kỹ thuật gộp đặc trưng được áp dụng trong những hướng tiếp cận đầu tiên cho bài toán, đạt được kết quả có tiềm năng khi R-MAC \cite{imageSearchKernel} tận dụng được điểm mạnh của cả lớp max pooling và sum pooling nhưng không đạt được hiệu năng bằng kỹ thuật tổng hợp đặc trưng NetVLAD \cite{arandjelovic2016netvlad}. Phần lớn các mô hình đạt hiệu quả cao nhất đều sử dụng kỹ thuật tổng hợp đặc trưng NetVLAD hoặc một biến thể tương tự sử dụng bổ sung các cơ chế như attention, semantics, \dots để tăng cường độ bền của đặc trưng. Gần đây, mô hình CosPlace \cite{berton2022rethinking} đạt được hiệu năng nổi bật với tác động tài nguyên thấp khi sử dụng lớp tổng hợp đặc trưng mới GeM \cite{GeM} và hàm mất mát ArcFace \cite{Deng_2022} để huấn luyện mô hình dưới góc nhìn của một bài toán phân loại. Với sự phát triển mới của các mô hình đẳng cực (isometric), cơ chế tự tập trung (self-attention) không còn được đánh giá là yếu tố chủ chốt cho hiệu năng của các mô hình sử dụng cơ chế tập trung thị giác (Vision transformers). Phát hiện này đã mở ra hướng phát triển mới cho mô hình mạng nơ-ron truyền thẳng nhiều lớp tích hợp cấu trúc trộn đặc trưng (MLP-Mixer). Lấy cảm hứng và cải tiến từ mô hình TransVPR \cite{wang2022transvpr}, Tolstikhin và nnk đề xuất mô hình MixVPR \cite{alibey2023mixvpr}, sử dụng mô hình MLP-Mixer cho tác vụ tổng hợp đặc trưng và đạt kết quả với tiềm năng phát triển cao. Cùng lúc đó, Keetha và nnk đề xuất \cite{keetha2023anyloc}, một mô hình học truy xuất địa danh sử dụng mô đun trích đặc trưng DINOv2, một mô hình trích đặc trưng được huấn luyện trên tập dữ liệu rất lớn, đa miền với mục đích hoạt động như mô hình cơ sở (backbone) cho các mô hình khác và mô đun tổng hợp đặc trưng VLAD. Mô hình này đạt kết quả tốt nhất hiện tại trên rất nhiều miền hình ảnh khác nhau như ảnh trong nhà, ảnh thành thị, ảnh thiên nhiên, ảnh ngoài trời dưới các điều kiện thời gian khác nhau, góc nhìn khác nhau, \dots Tuy nhiên, khi so sánh tập dữ liệu thành thị, mô hình MixVPR vẫn đạt kết quả cạnh tranh với tác động tài nguyên huấn luyện và tài nguyên cần thiết để chạy thấp hơn. Vì lý do này, chúng tôi nhận định MixVPR là mô hình có tiềm năng phát triển cao trong tương lai gần.
