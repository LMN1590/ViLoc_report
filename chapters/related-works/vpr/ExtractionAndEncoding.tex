\subsection{Trích xuất và biểu diễn đặc trưng}

Dưới góc nhìn một bài toán trích truy vấn ảnh, các đặc trưng được trích xuất từ ảnh lưu trữ trong cơ sở dữ liệu sẽ mang tính quyết định đối với chất lượng và hiệu năng của cả mô hình. Để đảm bảo tính đặc trưng cho các ảnh ở các địa danh khác nhau, quá trình trích xuất đặc trưng, ban đầu, được thiết kế một cách thủ công, đến từ kiến thức chuyên môn. Từ đó, đặt ra các định nghĩa về các loại đặc trưng: đặc trưng lân cận, đặc trưng toàn cục và độ quan trọng của chúng. Thời gian gần đây, với sự phát triển của thị giác máy tính, các phương pháp học biểu diễn bằng mạng nơ-ron học sâu cũng đạt được các thành tựu nổi bật, đặc biệt là mạng nơ-ron tích chập. Các hướng tiếp cận này, tuy chưa đạt được hiệu năng như các phương pháp định vị trực quan dùng dữ liệu ba chiều, nhưng đã thu hẹp đáng kể khoảng cách giữa chúng và phát huy được tính dễ phát triển và khả năng áp dụng cho quy mô lớn.

\subsubsection{Biễu diễn đặc trưng lân cận  - Local descriptors}

Trong bài toán học biễu diễn đặc trưng, một đặc trưng lân cận chỉ phân tích, biểu diễn ý nghĩa của một nhóm các điểm ảnh nhỏ và chỉ ra sự khác biệt giữa chúng với các nhóm các điểm ảnh lân cận\cite{CGV-017-localdescriptors}. Các nhóm điểm ảnh được tạo ra bằng cách chia mỗi hình ảnh thành lưới với độ phân giải nhất định và tất cả các nhóm điểm ảnh được thu thập, trích xuất để tìm đặc trưng. Để áp dụng cho bài toán truy xuất hình ảnh trực quan, các thuật toán lựa chọn như Hessian-Affine detector\cite{hessian-affine-detector}, MSER\cite{MSER-detector} được áp dụng để lựa chọn chỉ những nhóm điểm ảnh mang tính quan trọng, quyết định đến sự khác nhau giữa các ảnh và loại bỏ các nhóm điểm ảnh dễ bị ảnh hưởng bởi yếu tố môi trường. Vào thời gian đầu, việc truy xuất đặc trưng từ các nhóm ảnh, điểm ảnh được thực hiện bằng cách sử dụng các thuật toán như SIFT\cite{lowe1999object}, SURF\cite{bay2006surf}, RootSIFT\cite{Arandjelovi2012ThreeTE}, BRIEF\cite{brief}, DSP-SIFT\cite{Dong2014DomainsizePI} và đặc trưng kernel\cite{kernel-descriptors}.

Các hướng tiếp cận mới hơn cho rằng các hướng tiếp cận trên sử dụng số lượng lớn đặc trưng cần trích xuất và lưu trữ trong khi không phải tất cả các đặc trưng đều có đủ tính phân biệt cho quá trình truy xuất ảnh \cite{predicting-good-features}, đề xuất thêm quá trình lọc đặc trưng vào quá trình trích xuất.

Năm 2014, Jégou và Zisserman chuẩn hóa quá trình trích xuất véc-tơ biểu diễn từ đặc trưng cục bộ thành quy trình 2 bước\cite{Jegou_2014_CVPR}:

\begin{enumerate}
    \item bước nhúng chuyển mỗi véc-tơ đặc trưng lên không gian mới có số chiều cao hơn,
    \item bước tổng hợp tạo chỉ một véc-tơ biểu diễn từ các véc-tơ đã tạo.
\end{enumerate}

Năm 2015, \cite{selective-match-kernel} đề xuất hàng loạt các phương án cho các bước nhúng, tổng hợp, cùng với các giải pháp để lựa chọn mức độ đóng góp của mỗi cặp biểu diễn cho một nhóm các ảnh, đánh dấu nền móng về hướng phát triển và hiện thực cho các mô hình học biễu diễn đặc trưng cục bộ sau này.

\subsubsection{Biễu diễn đặc trưng toàn cục  - Global descriptors}

Trong khi việc tổng hợp các véc-tơ biểu diễn cục bộ hỗ trợ việc truy xuất một véc-tơ biễu diễn cho một hình ảnh, các véc-tơ biểu diễn toàn cục có thể truy xuất tất cả thông tin này một các trực tiếp. Khác với véc-tơ trích xuất cục bộ, các véc-tơ trích xuất toàn cục đọc và trích xuất biểu diễn đặc trưng cho toàn bộ bức ảnh và chỉ ra các đặc điểm khác biệt giữa bức ảnh này và bức ảnh khác. Việc truy xuất véc-tơ toàn cục có thể được hiện thực và thực thi với yêu cầu tài nguyên ít hơn sơ với việc truy xuất véc-tơ cục bộ, nhưng đồng thời cũng giảm đi khả năng phân biệt và tính bền bỉ trước các yếu tố thay đổi do điều kiện môi trường, góc chụp và các yếu tố bị che khuất. Mặc dù vậy, lợi ích của việc trích xuất đặc trưng cục bộ, áp dụng bởi các phương pháp như HOG\cite{HOG}, Gist\cite{GIST}, vẫn được xem trọng trong quá trình trích xuất biểu diễn đặc trưng toàn cục và trở thành cảm hứng cho các mô hình học biểu diễn sử dụng mạng nơ-ron học sâu sau này.

\subsubsection{Học biễu diễn bằng mạng nơ-ron học sâu tích chập và các lớp kết nối đầy đủ}

Với sự thành công của mạng nơ-ron học sâu trong lĩnh vực thị giác máy tính\cite{krizhevsky2012imagenet}, các mô hình học sâu này đã được công nhận là những phương pháp tạo biểu diễn vô cùng hiệu quả. Đồng thời, một số bài báo cũng đã chỉ ra khả năng học đặc trưng thường thấy và chuyển tiếp được (transferable) đến các bài toán khác\cite{Oquab_2014_CVPR,ZeilerVisualizingAU,Chen2018DeepLabSI}.

Các mô hình học sâu tích chập đầu tiên được sử dụng cho bài toán truy xuất hình ảnh trực quan được cấu hình từ các lớp kết nối đầy đủ\cite{razavian2014cnn, gong2014multiscale, babenko2014neural, deepindex, image-classification-retrieval, wang2014deep} của một mạng phân loại, được huấn luyện trên tập dữ liệu ImageNet\cite{russakovsky2015imagenet}, đạt hiệu quả đáng kể khi sử dụng hàm mất mát triplet\cite{wang2014deep, gomezojeda2015training}. Tuy nhiên, có thể sớm nhận thấy rằng các mô hình sử dụng nhiều lớp kết nối đầy đủ có ý nghĩa tương ứng với các biểu diễn đặc trưng toàn cục. Các biểu diễn đặc trưng tạo ra được bằng cách sử dụng phương pháp này thiếu độ bền đối với dữ liệu chứa nhiều yếu tố gây nhiễu, che khuất và không đủ các yếu tố bất biến đối với biến thể tịnh tiến và tỉ lệ. Để khắc phục các hạn chế này, nhiều mô hình học sâu đã đề xuất các phương án cắt ảnh thành nhiều nhóm điểm ảnh và huấn luyện nhiều biểu diễn với kết nối đầy đủ cho mỗi hình\cite{razavian2014cnn, babenko2014neural}. Khi đánh giá các cách tiếp cận này so với các phương pháp trích xuất đặc trưng thủ công cục bộ, các mô hình học sâu sử dụng ít bộ nhớ hơn nhưng yêu cầu số lượng tính toán cao hơn. Các biểu diễn kết nối đầy đủ bị giới hạn bởi kích thước đầu vào cố định và lượng lớn các tham số.

\subsubsection{Học biểu diễn bằng mạng nơ-ron tích chập}

Nỗ lực tìm giải pháp cho các giới hạn của mạng nơ-ron kết nối đầy đủ đã trở thành cảm hứng cho các phương pháp sử dụng trực tiếp kết quả nhận được từ các lớp tích chập của mạng nơ-ron học sâu. Áp dụng hướng tiếp cận đầu tiên chính là bài nghiên cứu của Babenko và nnk\cite{babenko2014neural}. Mạng nơ-ron tích chập tạo ra một tenxơ (tensor) có cấu hình \(C \times H \times W\), trong đó \(C\) là số kênh, \(H\) và \(W\) là chiều cao và chiều rộng của ảnh. Babenko và nnk tạo ra véc-tơ biểu diễn bằng cách làm phẳn và chuẩn hóa lớp \(C \times H \times W\). Một số bài báo lấy cảm hứng từ hướng nghiên cứu này\cite{hou2015convolutional}, sau đó, đã thể hiện rằng véc-tơ biểu diễn mạng nơ-ron học được, tùy vào phương án kết hợp lớp tenxơ, có thể mang ý nghĩa và chức năng tương tự với biểu diễn đặc trưng toàn cục trong khi tránh được các hạn chế của các mô hình học sâu sử dụng lớp kết nối đầy đủ. Các giải pháp này có thể được phân loại vào hai nhóm chính\cite{Masone2021ASO}:

\begin{itemize}
    \item tổng hợp (aggregation) đặc trưng tích chập bằng các giải pháp lấy cảm hứng từ trích xuất đặc trưng thủ công cục bộ;
    \item gộp (pooling) đặc trưng bằng cách khái quát hóa các đặc trưng tích chập.
\end{itemize}