\subsection{Trích xuất và biểu diễn đặc trưng}

Dưới góc nhìn một bài toán trích truy vấn ảnh, các đặc trưng được trích xuất từ ảnh lưu trữ trong cơ sở dữ liệu sẽ mang tính quyết định đối với chất lượng và hiệu năng của cả mô hình. Để đảm bảo tính đặc trưng cho các ảnh ở các địa danh khác nhau, quá trình trích xuất đặc trưng, ban đầu, được thiết kế một cách thủ công, đến từ kiến thức chuyên môn. Từ đó, đặt ra các định nghĩa về các loại đặc trưng: đặc trưng lân cận, đặc trưng toàn cục và độ quan trọng của chúng. Thời gian gần đây, với sự phát triển của thị giác máy tính, các phương pháp học biểu diễn bằng mạng nơ-ron học sâu cũng đạt được các thành tựu nổi bật, đặc biệt là mạng nơ-ron tích chập. Các hướng tiếp cận này, tuy chưa đạt được hiệu năng như các phương pháp định vị trực quan dùng dữ liệu ba chiều, nhưng đã thu hẹp đáng kể khoảng cách giữa chúng và phát huy được tính dễ phát triển và khả năng áp dụng cho quy mô lớn.

\subsubsection{Biễu diễn đặc trưng lân cận  - Local descriptors}

Trong bài toán học biễu diễn đặc trưng, một đặc trưng lân cận chỉ phân tích, biểu diễn ý nghĩa của một nhóm các điểm ảnh nhỏ và chỉ ra sự khác biệt giữa chúng với các nhóm các điểm ảnh lân cận\cite{CGV-017-localdescriptors}. Các nhóm điểm ảnh được tạo ra bằng cách chia mỗi hình ảnh thành lưới với độ phân giải nhất định và tất cả các nhóm điểm ảnh được thu thập, trích xuất để tìm đặc trưng. Để áp dụng cho bài toán truy xuất hình ảnh trực quan, các thuật toán lựa chọn như Hessian-Affine detector\cite{hessian-affine-detector}, MSER\cite{MSER-detector} được áp dụng để lựa chọn chỉ những nhóm điểm ảnh mang tính quan trọng, quyết định đến sự khác nhau giữa các ảnh và loại bỏ các nhóm điểm ảnh dễ bị ảnh hưởng bởi yếu tố môi trường. Vào thời gian đầu, việc truy xuất đặc trưng từ các nhóm ảnh, điểm ảnh được thực hiện bằng cách sử dụng các thuật toán như SIFT\cite{lowe1999object}, SURF\cite{bay2006surf}, RootSIFT\cite{Arandjelovi2012ThreeTE}, BRIEF\cite{brief}, DSP-SIFT\cite{Dong2014DomainsizePI} và đặc trưng kernel\cite{kernel-descriptors}.

Các hướng tiếp cận mới hơn cho rằng các hướng tiếp cận trên sử dụng số lượng lớn đặc trưng cần trích xuất và lưu trữ trong khi không phải tất cả các đặc trưng đều có đủ tính phân biệt cho quá trình truy xuất ảnh \cite{predicting-good-features}, đề xuất thêm quá trình lọc đặc trưng vào quá trình trích xuất.

Năm 2014, Jégou và Zisserman chuẩn hóa quá trình trích xuất véc-tơ biểu diễn từ đặc trưng cục bộ thành quy trình 2 bước\cite{Jegou_2014_CVPR}:

\begin{enumerate}
    \item bước nhúng chuyển mỗi véc-tơ đặc trưng lên không gian mới có số chiều cao hơn,
    \item bước tổng hợp tạo chỉ một véc-tơ biểu diễn từ các véc-tơ đã tạo.
\end{enumerate}

Năm 2015, \cite{selective-match-kernel} đề xuất hàng loạt các phương án cho các bước nhúng, tổng hợp, cùng với các giải pháp để lựa chọn mức độ đóng góp của mỗi cặp biểu diễn cho một nhóm các ảnh, đánh dấu nền móng về hướng phát triển và hiện thực cho các mô hình học biễu diễn đặc trưng cục bộ sau này.

\subsubsection{Biễu diễn đặc trưng toàn cục  - Local descriptors}

\subsubsection{Học biễu diễn bằng mạng nơ-ron học sâu}

\subsubsection{Học biễu diễn bằng mạng nơ-ron tích chập}
