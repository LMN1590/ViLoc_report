\subsection{Tìm kiếm tương đồng}
Bước thứ hai trong quy trình truy xuất ảnh thường là tác vụ tìm kiếm k lân cận gần nhất, hay nói là cách khác là tìm ra k ảnh có sự tương đồng cao nhất với ảnh nhận vào từ cơ sở dữ liệu. Tuy đơn giản về bản chất nhưng đây lại là tác vụ cực kỳ tốn kém về mặt tài nguyên đặc biệt là trong những bài toán với số chiều lớn. Một số nghiên cứu đã tăng tốc quá trình tìm kiếm bằng việc sử dụng các phương pháp tìm kiếm lân cận xấp xỉ (ANN) - không tiến hành quét cạn, sử dụng các kiến trúc chỉ mục khác,... \cite{4270175, Xie2015ImageCA, Mikolajczyk2007ImprovingDF, Muja2009FastAN, Muja2012FastMO, wang2017survey, magliani2019efficient, johnson2017billionscale}. 

Với bước truy xuất ảnh, một yếu tố quan trọng cần cân nhắc chính là điều kiện bộ nhớ cần thiết của các phương pháp tìm kiếm - các biểu diễn ảnh quá lớn sẽ dẫn đến tiêu tốn cực nhiều tài nguyên của cơ sở dữ liệu. Để giải quyết vấn đề này, các nhà nghiên cứu có xu hướng hướng đến việc cải thiện tác vụ tìm kiếm tương đồng về mặt ổn định cũng như khả năng mở rộng. Các cấu trúc chỉ mục dựa trên tệp đảo ngược \cite{Salton1988TermWeightingAI} đã được sử dụng để thực hiện tìm kiếm không quét cạn, đặc biệt hiệu quả với các biểu diễn véc-tơ thưa \cite{Johns2011FromIT, Sivic2003VideoGA, imageSearchKernel, Mohedano2016BagsOL, noh2018largescale, Philbin2007ObjectRW}. Trong \cite{Johns2011FromIT}, thời gian truy xuất được giảm bằng cách nhóm các hình ảnh cơ sở dữ liệu tương tự và sau đó thực hiện việc ghép tương đồng theo cụm. Các kỹ thuật lượng tử hóa như k-means \cite{Philbin2007ObjectRW, Torii2013VisualPR}, nhị phân hóa \cite{Perronnin2010LargescaleIR} và lượng tử hóa tích \cite{imageSearchKernel} cũng được áp dụng để giảm yêu cầu bộ nhớ lưu trữ dữ liệu, các kỹ thuật này còn được kết hợp với tính toán khoảng cách không đối xứng \cite{5432202} và phép gán bội \cite{imageSearchKernel, Jgou2008HammingEA, 5432202, Philbin2007ObjectRW, Tolias2014VisualQE, Li2015PairwiseGM} để giảm thiểu tối đa lỗi lượng tử hóa. Kỹ thuật chỉ mục đảo cũng đã được tổng quát hóa để làm việc với lượng tử hóa tích \cite{Guo2016DeepLF, 5432202, Babenko2012TheIM}, cải thiện thêm tốc độ và độ chính xác của phép tìm kiếm, với một chi phí bộ nhớ tương đối nhỏ.