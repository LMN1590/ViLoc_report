\chapter{GIỚI THIỆU}

\section{Động cơ nghiên cứu}

\subsection{Nhu cầu sử dụng trong thực tế}

Việc có thể nhận biết được môi trường xung quanh là một tác vụ mang tính cốt lõi trong các lĩnh vực công nghệ tương tác thực tế ảo \cite{middelberg2014scalable}, định hướng \cite{sarlin2023orienternet}, xe tự hành \cite{chaabane2021end}, robot \cite{sunderhauf2015place}, \dots Sự phối hợp của hệ thống la bàn và định vị GPS là một phương pháp truyền thống được sử dụng rộng rãi ngày nay để xác định được tư thế của một thiết bị (bao gồm vị trí và góc quay). Tuy nhiên, trong các điều kiện hoạt động hiện đại, vì phải tương tác với hệ thống các vệ tinh, hệ thống định vị GPS xuất hiện nhiều yếu điểm khó khắc phục. Công nghệ này phụ thuộc hoàn toàn vào chất lượng của liên kết giữa thiết bị và các vệ tinh. Khi hoạt động trong môi trường trong nhà và môi trường thành thị, bao quanh bởi nhiều toà nhà và thiết bị vô tuyến, liên kết này gặp nhiều yếu tố gây nhiễu và trở nên thiếu tin cậy. Vì lý do này, để đáp ứng nhu cầu định vị thiết bị một cách chính xác và đáng tin cậy trong nhiều khu vực và điều kiện môi trường khác nhau, bài toán định vị trực quan - Visual Localization, đã được phát triển từ lĩnh vực thị giác máy tính.

Bài toán Visual Localization hướng đến việc định hướng được tư thế (vị trí và góc quay) của một thiết bị trong không gian khi chỉ nhận được thông tin thị giác như hình ảnh, chuỗi hình ảnh, \dots. Trong những năm gần đây, lĩnh vực định vị trực quan đã có những bước phát triển đáng kể khi ứng dụng những mô hình học sâu, thể hiện qua lượng lớn bài báo nghiên cứu khoa học hàng năm về chủ đề này. Phần lớn các bài báo đã đưa ra nhiều hướng tiếp cận đa dạng để giải quyết bài toán định vị trực quan, với sự tập trung chủ yếu vào việc giải quyết hai vấn đề chính là:
\begin{itemize}
  \item Tìm kiếm một cách biểu diễn môi trường xung quanh tuy đơn giản, nhưng vẫn đảm bảo được tính hiệu quả trong việc truy xuất thông tin nhằm tiết kiệm tài nguyên để xây dựng, duy trì, mở rộng và sử dụng.
  \item Cải thiện độ chính xác của tư thế được dự đoán mà vẫn đảm bảo được khả năng tổng quát hóa của mô hình trên nhiều môi trường khác nhau.
\end{itemize}

\subsection{Những hướng tiếp cận đã được đề xuất}

Với cùng mục tiêu là xác định được tư thế của thiết bị chụp ảnh, nhiều phương pháp khác nhau đã được đề xuất để giải quyết bài toán định vị trực quan. Một nhóm phương pháp truyền thống mà cho đến hiện tại vẫn cho ra kết quả cạnh tranh hoạt động dựa trên cấu trúc của cảnh là structure-based. Phương pháp này mô phỏng lại cấu trúc thật của môi trường đang xét bằng các tập hợp điểm trong không gian, tạo thành các đám mây điểm ba chiều với mỗi điểm chứa mô tả trực quan của khu vực tương ứng. Từ đó, tư thế chính xác của ảnh chụp có thể được xác định thông qua việc tìm kiếm những điểm tương đồng giữa ảnh truy vấn hai chiều và bản đồ đám mây ba chiều.

Một hướng đi khác, thiên về việc ứng dụng kiến trúc học sâu đầu-cuối (end-to-end) vào việc tính toán, gọi là hồi quy tư thế máy ảnh, tập trung vào việc xác định tư thế của thiết bị chụp ảnh truy vấn từ những ảnh tham khảo có độ tương đồng cao đã được gắn nhãn với tư thế chính xác. Hồi quy tư thế máy ảnh chia thành hai hướng tiếp cận chính: hồi quy tương đối và hồi quy tuyệt đối. Với phương pháp hồi quy tương đối, từ cặp ảnh gồm ảnh truy vấn và ảnh tham khảo, mô hình xác định được sai số về tư thế giữa hai ảnh \cite{zhou2020learn} để hồi quy được tư thế tuyệt đối. Đối với phương pháp hồi quy tư thế tuyệt đối, thay vì truy xuất ảnh làm điểm mốc để xác định vị trí, phương pháp này xây dựng nội hàm cách biểu diễn của môi trường thông qua các trọng số của một mô hình học sâu và hồi quy trực tiếp tư thế tuyệt đối trong cùng một bước \cite{kendall2016posenet}.

Tác vụ truy xuất ảnh đóng vai trò quan trọng trong việc cung cấp một cột mốc cụ thể cho các mô hình hồi quy tương đối. Tuy nhiên, khi tách bài toán ra khỏi quá trình xác định tư thế tuyệt đối, các nỗi lực nghiên cứu đã gặt hái được nhiều thành công, đẩy mạnh phạm vi hoạt động của bài toán từ các khu vực nhỏ, khép kín đến các khu vực thành thị và thế giới \cite{berton2022rethinking, keetha2023anyloc, alibey2023mixvpr}.

\section{Mục tiêu đề tài}

Nhận thấy dù những nghiên cứu gần đây đã khám phá thêm nhiều hướng giải quyết khác nhau, đa phần những phương pháp này đều là những biến thể của ba hướng tiếp cận chính. Vì lý do này, các mô hình đều thừa hưởng được những điểm mạnh cũng như những điểm bất lợi cơ bản của các hướng tiếp cận tương ứng. Đối với nhóm những phương pháp truyền thống Structure-from-Motion, mô hình đa phần đều phụ thuộc vào bản đồ đám mây điểm 3D để hoạt động, một cấu trúc dữ liệu có yêu cầu tài nguyên rất cao để sử dụng. Nhóm phương pháp hồi quy tư thế máy ảnh bị giới hạn trong một phạm vi nhỏ và độ chính xác của các mô hình vẫn chưa thể so sánh được với nhóm phương pháp tái tạo cấu trúc truyền thống. Những phương pháp nhận diện địa điểm trực quan cho thấy tiềm năng đưa bài toán định vị qua hình ảnh lên một phạm vi rộng và bao quát hơn, đi kèm với sự đánh đổi về độ chính xác của mô hình.

Để phát triển bài toán định vị trực quan, nhóm đã đặt ra mục tiêu: xây dựng một pipeline định vị trực quan hoạt động hiệu quả trong phạm vi rộng khi được cung cấp một cơ sở dữ liệu chứa ảnh mô tả khu vực. Pipeline thực hiện hai quy trình chính một cách lần lượt: nhận diện địa điểm trực quan và ước tính tư thế máy ảnh bằng phương pháp hồi quy tư thế tương đối. Tác vụ nhận diện địa điểm trực quan giúp đưa phạm vi hoạt động của bài toán định vị lớn hơn, và giúp đưa ra ảnh tham khảo để làm điểm mốc cho quá trình hồi quy sau này. Tác vụ hồi quy tư thế tương đối giúp dự đoán được kết quả định vị chính xác từ kết quả thô là ảnh tham khảo trước đó. Chúng tôi cho rằng việc kết hợp hai tác vụ lại có thể giúp chúng bổ trợ cho những khiếm khuyết của nhau. Ngoài ra, mô hình sử dụng cho tác vụ nhận diện địa điểm trực quan sẽ được finetune dựa trên một phương pháp mining mới, nhằm hướng mô hình đến việc truy xuất những ảnh không những có vị trí chụp gần với ảnh truy vấn mà còn chia sẻ góc nhìn, được xác định bằng độ trùng lấp về vùng khung hình (frustum) giữa hai ảnh.

Chúng tôi tập trung đánh giá khả năng hoạt động của pipeline trên những tập dữ liệu được thu thập ở môi trường thành thị tại các khu vực nhỏ ngoài trời và những nơi mà hiệu quả của GPS bị hạn chế. Pipeline được cung cấp một tập dữ liệu tham khảo chứa những ảnh hai chiều thể hiện khu vực cần định vị. Đồng thời, ảnh trong tập dữ liệu được gắn nhãn tương ứng với tư thế chụp, gồm vị trí và góc quay của thiết bị đã sử dụng. Để thể hiện khả năng tổng quát hóa, thích ứng với môi trường mới và độ tối ưu hóa về mặt tài nguyên và thời gian, tập dữ liệu huấn luyện của các module tương ứng trong pipeline không chứa cùng thành phố hay địa danh với tập dữ liệu đánh giá. Hiệu quả của mô hình được đánh giá bằng độ lệch trung vị của vị trí và góc quay, với đơn vị lần lượt là (mét, độ).

\begin{itemize}
  \item \textbf{Dữ liệu đầu vào và đầu ra}
        \begin{itemize}
          \item Trong quá trình hoạt động, dữ liệu đầu vào của mô hình là một ảnh truy vấn do người dùng chụp được và đưa vào mô hình để xử lý. Đồng thời, một tập dữ liệu gồm các ảnh hai chiều được gắn nhãn tư thế, biểu diễn khu vực hỗ trợ được cung cấp cho mô hình. Những thông số intrinsics của ảnh chụp gồm tiêu cự và điểm chính của máy ảnh đều phải được lưu cùng với ảnh.
          \item Dữ liệu đầu ra bao gồm hai vector $R$ và $t$, lần lượt thể hiện vị trí và góc quay của thiết bị đã chụp ảnh trong không gian. Vector $R$ thể hiện góc quay có định dạng của một quaternion, $(q_w,q_x,q_y,q_z)$. Vector $t$ thể hiện vị trí của chụp ảnh trong không gian 3 chiều, có định dạng $(t_x,t_y,t_z)$.
        \end{itemize}
  \item \textbf{Kết quả mong đợi}
        \begin{itemize}
          \item Với mô hình kết hợp hai bước được đề xuất, chúng tôi kỳ vọng đạt được kết quả cạnh tranh hoặc vượt trội hơn so với các mô hình đơn lẻ khác trong tác vụ định vị trực quan ở những khu vực thành thị. Mô hình được kỳ vọng hoạt động tốt mà không cần tiêu tốn tài nguyên huấn luyện lại.
          \item Với miner đã thiết kế, chúng tôi kỳ vọng kết quả của module nhận diện địa điểm trực quan giúp chọn được những hình có độ trùng lấp khung hình cao, giúp cho tác vụ xác định tư thế máy ảnh của module hồi quy tư thế tương đối hoạt động hiệu quả hơn.
        \end{itemize}
\end{itemize}

\section{Kết luận}
Việc xác định chính xác tư thế của một vật trong không gian có phạm vi ứng dụng vô cùng lớn, là công nghệ chủ chốt trong rất nhiều lĩnh vực khác nhau. Tuy nhiên, việc chỉ dựa vào hệ thống không đảm bảo được sự ổn định như GPS sẽ hạn chế khả năng phát triển trong tương lai của các công nghệ này. Vậy nên bài toán định vị hóa trực quan đã phát triển một hệ thống định vị chính xác và ổn định hơn, dựa vào thông tin hình ảnh môi trường xung quanh.

Trong những phương pháp được đưa ra, chúng tôi tập trung vào việc kết hợp hai bài toán riêng lẻ là nhận diện địa điểm trực quan và hồi quy tư thế máy ảnh tương đối. Vì mỗi phương pháp có những lợi thế cũng như những vấn đề cần giải quyết riêng, qua việc ứng dụng và kết hợp cả hai bài toán trong một mô hình, chúng tôi hi vọng cả hai tác vụ này có thể bổ trợ, giải quyết điểm yếu của nhau.

Để có thể lựa chọn được những giải pháp phù hợp, chúng tôi đã tiến hành tìm hiểu, nghiên cứu về những kiến thức nền tảng đồng thời khảo sát những công trình nghiên cứu đã được công bố trước đây. Những nội dung này sẽ được thể hiện trong \textbf{Chương 2: Các công trình liên quan}. Đồng thời, giải pháp đề xuất của chúng tôi cũng sẽ được trình bày kỹ hơn ở \textbf{Chương 3: Phương pháp đề xuất}.
