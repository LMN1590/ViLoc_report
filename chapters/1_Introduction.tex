\chapter{GIỚI THIỆU}

\textit{Nội dung chương 1 sẽ đề cập đến nội dung của bài toán bản địa hóa trực quan - Visual Localization, những giải pháp đã được đề xuất hiện nay của bài toán, và những hạn chế của chúng. Từ đó, nhóm sẽ xác định mục tiêu cần thực hiện và phạm vi của đề tài}

\section{Động cơ nghiên cứu}

\subsection*{Nhu cầu sử dụng trong thực tế}

Theo thời gian, công nghệ ngày càng được tích hợp vào đời sống hàng ngày của con người. Sự xuất hiện của chúng không còn bị giới hạn ở một không gian cụ thể nào, mà dần tiến tới việc tương tác với không gian ngoài trời. Những công nghệ này bao gồm xe tự hành \cite{chaabane2021end}, robot \cite{sunderhauf2015place}, công nghệ tương tác thực tế ảo \cite{middelberg2014scalable}, định hướng \cite{sarlin2023orienternet}, ... Việc có thể nhận biết được môi trường xung quanh để có thể tương tác là một tác vụ cốt lõi trong những công nghệ trên và là yếu tố quyết định để đạt được hiệu quả tốt. 

Trước đây, những hệ thống định vị toàn cầu như GPS đã được sử dụng để xác định thông tin về môi trường. Tuy nhiên, những hệ thống này có một số khiếm khuyết như độ chính xác chỉ nằm trong khoảng vài mét, hiệu quả bị giới hạn ở không gian bên trong và thiếu thông tin về hướng quay nếu không sử dụng thêm la bàn. Với nhu cầu ngày càng tăng về độ chính xác, những hệ thống định vị toàn cầu dần trở nên không phù hợp. Từ đó, dẫn đến sự ra đời của bài toán bản địa hóa trực quan - Visual Localization - trong lĩnh vực thị giác máy tính.

Bộ não con người có thể thực hiện bài toán bản địa hóa trực quan bằng trực giác. Thông tin trực quan về môi trường nhận được bởi mắt sẽ được dùng để truy xuất thông tin từ biểu diễn bên trong bộ não con người về những nơi đã được đi qua. Tuy nhiên, để mô phỏng lại quá trình này, những giải thuật phức tạp liên quan đến việc xây dựng một cách biểu diễn phù hợp cho không gian như 3D-point cloud hay thực hiện feature matching đã được sử dụng. Những tác vụ này sẽ tiêu tốn rất nhiều tài nguyên để xây dựng và thực hiện.

Trong những năm gần đây, lĩnh vực này đã có những bước phát triển đáng kể, được thể hiện qua một lượng lớn bài báo nghiên cứu khoa học. Những bài báo này đã đưa ra những hướng đi đa dạng để giải quyết bài toán bản địa hóa trực quan, nhưng đa số đều tập trung vào hai chủ đề chính là:
\begin{itemize}
    \item Tìm kiếm một cách biểu diễn tuy đơn giản, nhưng vẫn đảm bảo được tính hiệu quả trong việc truy xuất thông tin nhằm tiết kiệm tài nguyên để xây dựng, duy trì, mở rộng và sử dụng.
    \item Cải thiện độ chính xác của vị trí được truy xuất mà vẫn đảm bảo được tính hiệu quả.
\end{itemize}

\subsection*{Những hướng đi đã được đề xuất}

Với mục tiêu là xác định được vị trí mà ảnh được chụp, nhiều phương pháp khác nhau đã được đề xuất để giải quyết bài toán bản địa hóa trực quan. Một nhóm phương pháp truyền thống mà cho đến hiện tại vẫn cho ra kết quả cạnh tranh là phương pháp dựa trên cấu trúc của cảnh - Structure-based Method. Phương pháp này sẽ dựa trên việc tái tạo lại cấu trúc của môi trường đang xét bằng một tập các điểm trong không gian 3D, tạo thành một 3D-point cloud và tiến hành xác định những cặp đặc trưng cục bộ tương ứng với nhau trong ảnh truy vấn và bản đồ 3D \cite{sattler2011fast}. Từ đó, tọa độ chính xác của vị trí chụp ảnh có thể được xác định. Để tối ưu hóa quá trình xác định tương ứng của các điểm ảnh, phương pháp truy xuất ảnh có thể được sử dụng để giới hạn lại không gian tìm kiếm. Chỉ những đặc trưng nằm trong những ảnh tương đồng truy xuất ra mới được xét \cite{sarlin2019coarse}. Ngoài ra, ở bước xác định những cặp đặc trưng tương ứng, thay vì sử dụng những phương pháp được định nghĩa sẵn bởi con người, mạng học sâu có thể được ứng dụng để trực tiếp xác định vị trí của các điểm ảnh trong không gian 3D \cite{brachmann2021visual}. 

Một hướng đi khác, sử dụng những ảnh có nét tương đồng truy xuất được, vị trí cuối cùng có thể được nội suy từ vị trí của những ảnh đó. Kết quả này sẽ chỉ là một ước tính sơ bộ với độ chính xác khá thấp \cite{pion2020benchmarking}. Với phương pháp hồi quy tương đối, từ cặp ảnh gồm ảnh truy vấn và ảnh tham chiếu, mô hình sẽ xác định được khoảng cách về vị trí giữa hai ảnh \cite{zhou2020learn}. Ngoài ra, thay vì truy xuất ảnh làm điểm mốc để xác định vị trí, phương pháp hồi quy vị trí tuyệt đối sẽ xây dựng cách biểu diễn của môi trường bên trong mô hình và có thể tính trực tiếp kết quả chỉ với đầu vào là ảnh truy vấn \cite{kendall2016posenet}.

\subsection*{Xác định những vấn đề hiện hữu trong bài toán bản địa hóa trực quan}

\begin{itemize}
    \item \textbf{Vấn đề với những phương pháp sử dụng biểu diễn 3D}
    \begin{itemize}
        \item Vấn đề về lưu trữ:
        \item Vấn đề về khả năng mở rộng của mô hình
        \item Vấn đề về khả năng khái quát hóa
    \end{itemize}
    \item \textbf{Vấn đề với những phương pháp hồi quy vị trí}
    \begin{itemize}
        \item Vấn đề về tài nguyên:
        \item Vấn đề về khả năng mở rộng của mô hình
        \item Vấn đề về khả năng khái quát hóa
    \end{itemize}
    \item \textbf{Những vấn đề khác}
    \begin{itemize}
        \item Vấn để duy trì tập dữ liệu:
    \end{itemize}
\end{itemize}

\begin{itemize}
    \item \textbf{Vấn đề lưu trữ}
    \begin{itemize}
        \item 3D-point cloud tốn quá nhiều bộ nhớ để duy trì, đặc biệt là với những thành phố lớn
        \item Có thể compress được nhưng mà sẽ đánh đổi bằng độ chính xác đặc trưng của phương pháp 2D-3D
    \end{itemize}
    
    \item \textbf{Vẫn đề tài nguyên}
    \begin{itemize}
        \item Những mô hình học sâu thường có xu hướng nặng về tài nguyên tính toán: các mô hình VLAD(do nó cần nhiều descriptor ~32768). Được nhắc đến trong CosPlace
        \item Những mô hình SfM nặng về việc lưu trữ data + do cần phải tạo bản đồ 3D để có thể bắt đầu tính toán
    \end{itemize}
    \item \textbf{Vấn đề về khả năng mở rộng của mô hình}
    \begin{itemize}
        \item Mô hình deep learning chỉ chạy tốt trên những dataset nhỏ như 7Scenes và Cambridge Landmark, không phải những tập dữ liệu rộng và thưa thớt như Aachen và Mapillary Street-level Sequences
    \end{itemize}
    \item \textbf{Vấn để duy trì tập dữ liệu}
    \begin{itemize}
        \item Ảnh vệ tinh do khó thu thập được nên đa số các ảnh vệ tinh sẽ bị lỗi thời không phản ánh đúng được tình hình hiện tại của môi trường. Ngoài ra, do ảnh vệ tinh chứa thông tin của một vùng rộng lớn nên khi có sự cập nhật của một khu vực nhỏ sẽ không được phản ánh hiệu quả trong ảnh vệ tinh.
        \item Do những tập dữ liệu thành thị bao phủ một khu vực rộng lớn và dày đặc các địa danh(tòa nhà, công viên, trường học,...) nên đa số các tập dữ liệu sẽ không thể nào chứa hết được những góc nhìn có thể có. Dẫn đến việc ảnh lấy được từ dữ liệu sẽ không có nhiều điểm tương đồng với ảnh truy vấn.
    \end{itemize}
    \item \textbf{Vấn đề về khả năng khái quát hóa}
    \begin{itemize}
        \item Phần lớn các mô hình chỉ hoạt động trong một loại địa hình khu vực như một thành phố cố định (CosPlace, MixVPR,...). Trong khi AnyLoc tuyên bố có thể hoạt động trên đa địa hình.
        \item Một số mô hình xây dựng cách biểu diễn của khu vực theo cách bao hàm bên trong mô hình học sâu (hồi quy vị trí tuyệt đối), hoặc trải qua quá trình tiền xử lý như xây dựng bản đồ đám mây điểm 3D cho nên khi cập nhật tập dữ liệu, sẽ cần phải xử lý lại. Vậy nên, sẽ cần trải qua quá trình huấn luyện hoặc xử lý lại khi cập nhật tập dữ liệu.
        \item Một số mô hình học sâu cho ảnh như CNN không thể nắm bắt được những đặc trưng hình học khi được huấn luyện trên những hàm loss đơn giản(for more information, please refer to the paper to learn or not to learn, EssNet)
    \end{itemize}
\end{itemize}

\section{Mục tiêu đề tài}

Nhóm đề xuất một kiến trúc mô hình gồm 2 phần: truy xuất ảnh và hồi quy vị trí máy ảnh. 
\begin{itemize}
    \item Truy xuất ảnh
    \item Hồi quy vị trí tương đối
\end{itemize}

\section{Phạm vi đề tài}
\begin{itemize}
    \item \textbf{Mục tiêu chung} 
    \begin{itemize}
        \item Hướng đến việc thiết kế và áp dụng một kiến trúc mô hình học sâu để giải quyết bài toán bản địa hóa trực quan.
    \end{itemize} 
    \item \textbf{Kết quả mong đợi}
    \begin{itemize}
        \item Đối với truy xuất ảnh, nhóm mong có thể giúp cải thiện độ chính xác của những mô hình truy xuất ảnh đã được đề xuất trước đây.
        \item Đối với mô hình hồi quy vị trí máy ảnh, nhóm mong có thể giúp mô hình được sử dụng trên các tập dữ liệu lớn hơn.
    \end{itemize}
    \item \textbf{Thời gian}
    \begin{itemize}
        \item Đồ án chuyên ngành kéo dài trong 15 tuần
        \item Đồ án tốt nghiệp kéo dài trong 15 tuần
    \end{itemize}
    \item \textbf{Lĩnh vực hướng đến}
    \begin{itemize}
        \item Đóng góp cho những nghiên cứu về bài toán bản địa hóa trực quan trong chuyên ngành thị giác máy tính.
    \end{itemize}
    
\end{itemize}

\section{Cấu trúc đồ án chuyên ngành}
Đồ án chuyên ngành sẽ bao gồm năm (6) chương, bao gồm cả chương này. Nội dung ủa mỗi chương như sau:
\begin{itemize}
    \item \textbf{Chương 1: Giới thiệu} \\
    Trình bày sơ lược về động cơ nghiên cứu, mục tiêu và phạm vi đề tài giải quyết.
    \item \textbf{Chương 2: Công trình liên quan} \\
    Chương này đề cập tới các kiến thức lý thuyết của các công trình nghiên cứu có liên quan.
    \item \textbf{Chương 3: Kiến thức nền tảng} \\
    Chương này đề cập tới các kiến thức lý thuyết nền tảng, các nguyên tắc cần được đảm bảo khi hiện thực và tiến hành so sánh, lựa chọn công nghệ sử dụng.
    \item \textbf{Chương 4: Phương pháp đề xuất} \\
    Chương này đề cập đến phương pháp giải quyết bài toán mà nhóm đề xuất bao gồm tổng quan về cơ chế cũng như lý thuyết cách hoạt động.
    \item \textbf{Chương 5: Kết quả khảo sát} \\
    Chương này đề cập kết quả khảo sát của nhóm bao gồm kết quả đánh giá hiệu quả của các phương pháp truy xuất ảnh và hồi quy vị trí máy ảnh.
    \item \textbf{Chương 6: Thảo luận} \\
    Chương này đề cập đến các câu hỏi cũng như các vấn đề nảy sinh trong quá trình nghiên cứu, thu thập kiến thức và hiện thực mô hình.
\end{itemize}
Cấu trúc của toàn đề tài sẽ được trình bày trong giai đoạn Đồ án tốt nghiệp.

\section{Kết chương}
Việc xác định chính xác vị trí có một phạm vi ứng dụng rộng rãi, là công nghệ chủ chốt trong rất nhiều lĩnh vực khác nhau. Tuy nhiên, việc chỉ dựa vào hệ thống không đảm bảo được sự ổn định như GPS sẽ hạn chế khả năng phát triển trong tương lai của những lĩnh vực ấy. Vậy nên bài toán định vị hóa trực quan đã được đưa ra để có một hệ thống đưa ra định vị chính xác và ổn định, dựa vào thông tin hình ảnh môi trường xung quanh.

Trong những phương pháp được đưa ra, nhóm tập trung vào việc kết hợp 2 hướng xử lý là truy xuất ảnh và hồi quy tương đối vị trí. Mỗi phương án sẽ có một vấn đề riêng. Đối với việc truy xuất ảnh, kết quả cho ra được sẽ không có độ chính xác cao, do vị trí của ảnh truy xuất được sẽ được lấy làm kết quả. Đối với hướng hồi quy tương đối vị trí, đa số những phương pháp trước đây đều tập trung vào việc hồi quy trong những không gian nhỏ, có lượng dữ liệu dày đặc, không phù hợp với tập dữ liệu đô thị, mục tiêu của bài nghiên cứu của nhóm. Qua việc ứng dụng cả 2 cách giải quyết trong một mô hình, nhóm hy vọng 2 mô hình có thể bổ trợ, giải quyết điểm yếu của nhau.

Để có thể lựa chọn được những giải pháp phù hợp, nhóm đã tiến hành khảo sát những kiến thức nền tảng và những công trình nghiên cứu đã được xuất bản trước đây. Những nội dung này sẽ được thể hiện trong \textbf{Chương 2: Các công trình liên quan}.