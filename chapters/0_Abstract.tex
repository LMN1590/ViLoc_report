\begin{abstract}

    Khả năng định vị toàn cầu có vai trò cốt lõi trong những lĩnh vực phụ thuộc vào việc nhận biết và tương tác với môi trường xung quanh như xe tự hành, robot và công nghệ thực tế ảo AR. Trước đây, những công nghệ này sẽ phụ thuộc vào những hệ thống định vị toàn cầu như GPS. Tuy nhiên, những hệ thống này vẫn còn những giới hạn nhất định. Vì vậy nên, bài toán định vị trực quan - Visual Localization - đã được đưa ra nhằm đạt được kết quả chất lượng hơn, thông qua dữ liệu trực quan thu được tại vị trí đó.

    Trong các các hướng tiếp cận hiện đại, nhận thấy các mô hình định vị trực quan cổ điển và định vị trực quan bằng phương pháp hồi quy trực tiếp thường đề xuất các mô hình có tác động tài nguyên lớn và khó phát triển về phạm vi, chúng tôi đặt ra mục tiêu nghiên cứu và sử dụng hướng tiếp cận dùng mô hình hai bước - nhận diện địa điểm trực quan (Visual Place Recognition) và ước tính vị trí của máy ảnh (Pose Estimation). Với những cải tiến liên tục trong những năm gần đây, các mô hình này đã đạt được hiệu quả cạnh tranh và tránh được tác động tài nguyên lớn, cho phép phạm vi bài toán được mở rộng lên mức độ thành thị và thế giới. Trong quá trị nhận dạng địa điểm trực quan, mô hình sẽ nhận vào một ảnh truy vấn và chọn ra một hay nhiều ảnh từ tập ảnh được cung cấp, thể hiện cùng một cảnh với ảnh đầu vào. Từ cặp ảnh truy vấn và ảnh tham khảo đầu vào, mô hình sẽ tính toán vị trí và hướng quay của máy ảnh trong không gian.

    Nhận thấy rằng hai hướng tiếp cận của bài toán khi đứng riêng đều đã có kết quả khả quan trong phạm vi của mình, và dữ liệu đầu ra và đầu vào của hai hướng tiếp cận tương thích với nhau, chúng tôi đặt ra mục tiêu nghiên cứu cấu trúc của mô hình và đề xuất phương hướng phát triển để kết hợp hai bước xử lý thành một quy trình hoàn chỉnh, nhằm bổ trợ cho những khiếm khuyết của mỗi hướng, sử dụng hai mô hình:

    \begin{itemize}
        \item MixVPR \cite{alibey2023mixvpr}, được phát triển bởi Ali-bey và nnk vào năm 2023 cho bước nhận diện địa điểm trực quan, hỗ trợ đưa mô hình định vị lên phạm vị không gian rộng lớn.
        \item Mô hình ước tính vị trí máy ảnh 2D-2D được đề xuất trong Map-free Relocalization \cite{arnold2022mapfree}, phát triển bởi Arnold và nnk vào năm 2022, hỗ trợ tính toán vị trí và tư thế chính xác của máy ảnh trong không gian từ mỗi cặp ảnh.
    \end{itemize}

\end{abstract}