\begin{abstract}

    Khả năng định vị toàn cầu đóng vai trò cốt lõi trong những lĩnh vực phụ thuộc vào việc nhận biết và tương tác với môi trường xung quanh như xe tự hành, robot và công nghệ thực tế ảo AR. Trước đây, những công nghệ này sẽ phụ thuộc vào những hệ thống định vị toàn cầu như GPS. Tuy nhiên, công nghệ này vẫn còn những giới hạn nhất định về độ chính xác vị trí cũng như không xác định được hướng xoay. Để đáp ứng nhu cầu ngày càng tăng về độ chuẩn xác, bài toán định vị trực quan - Visual Localization - đã ra đời với mục tiêu thông qua việc thu thập dữ liệu trực quan tại một khu vực nhất định, chúng ta có thể định vị chính xác vị trí cũng như hướng nhìn của một người tại khu vực đó.

	Sau quá trình khảo sát và nghiên cứu về các hướng tiếp cận bài toán định vị trực quan, chúng tôi nhận thấy hướng tiếp cận truy xuất ảnh, hay còn gọi là nhận diện địa điểm trực quan (Visual Place Recognition) mang nhiều tiềm năng khi có thể hoạt động trên phạm vi rộng mà không tiêu tốn nhiều tài nguyên tính toán. Đổi lại, nhóm phương pháp này thiếu đi khả năng tính toán tư thế (vị trí, góc quay) chính xác của đầu vào. Ngược lại, các mô hình thuộc nhóm phương pháp ước tính vị trí của máy ảnh tương đối (Relative Pose Estimation) tuy có thể tính toán chính xác tư thế tuyệt đối của ảnh, lại không hoạt động tốt trên phạm vi rộng. 

	Nhận thấy rằng mỗi bài toán khi hoạt động riêng lẻ đều có những điểm mạnh cũng như những hạn chế riêng, chúng tôi đặt ra mục tiêu nghiên cứu cấu trúc một mô hình có thể kết hợp hai bài toán trên (Visual Place Recognition và Relative Pose Estimation) thành một quy trình hoàn chỉnh, nhằm bổ trợ cho những khiếm khuyết của mỗi bài toán, với mục đích cuối cùng của chúng tôi là xây dựng một mô hình định vị trực quan hoạt động trên cả phạm vi nhỏ lẫn phạm vi rộng như thành thị. Cụ thể hơn, mô hình kết hợp của chúng tôi sử dụng hai mô hình:
    \begin{itemize}
        \item MixVPR \cite{alibey2023mixvpr}, được phát triển bởi Ali-bey và những cộng sự vào năm 2023, cho bước nhận diện địa điểm trực quan, hỗ trợ đưa mô hình định vị lên phạm vị không gian rộng lớn.
        \item Mô hình ước tính vị trí máy ảnh 2D-2D được đề xuất trong Map-free Relocalization \cite{arnold2022mapfree}, phát triển bởi Arnold và những cộng sự vào năm 2022, hỗ trợ tính toán vị trí và hướng quay chính xác của máy ảnh trong không gian từ mỗi cặp ảnh.
    \end{itemize}
Ngoài ra, với mô hình MixVPR \cite{alibey2023mixvpr} thuộc tác vụ Visual Place Recognition, chúng tôi đề xuất một chiến lược khai phá mới nhằm mục tiêu cố gắng nâng cao, cải thiện hiệu suất của tác vụ này trong mô hình kết hợp. Chi tiết sẽ được trình bày ở \textbf{Phương pháp đề xuất}. 
\end{abstract}