\begin{abstract}

Khả năng định vị toàn cầu có vai trò cốt lõi trong những lĩnh vực phụ thuộc nhiều vào việc nhận biết và tương tác phù hợp với môi trường xung quanh như xe tự hành, robot và công nghệ tương tác thực tế ảo AR. Trước đây, đa số những công nghệ này đều sẽ lấy thông tin về môi trường xung quanh nhờ vào những hệ thống định vị toàn cầu như GPS. Tuy nhiên, những hệ thống này cũng sẽ có những giới hạn nhất định về độ chính xác, độ tin cậy tùy thuộc vào khu vực được sử dụng. Vì vậy nên, bài toán bản địa hóa trực quan đã được đưa ra nhằm cung cấp thông tin đáng tin cậy hơn cho nhu cầu ngày càng tăng của quá trình phát triển của công nghệ.

Bản địa hóa trực quan - Visual Localization - là một bài toán hướng đến việc xây dựng một cách biểu diễn phù hợp và hiệu quả cho môi trường đang xét, từ đó có thể định vị một cách chính xác hơn trong môi trường đó. Bài toán sẽ chỉ quan tâm đến việc xử lý hình ảnh, video hoặc kết quả của những bộ cảm biến về môi trường xung quanh.

Hai hướng tiếp cận đã có những cải tiến liên tục trong những năm gần đây chính là hướng nhận dạng địa điểm trực quan - Visual Place Recognition - và hướng ước tính vị trí của máy ảnh - Pose Estimation. Với hướng tiếp cận nhận dạng địa điểm trực quan, mô hình sẽ nhận vào một ảnh truy vấn và chọn ra một hay nhiều ảnh từ tập ảnh được cung cấp sẵn mà được cho là thể hiện chung một cảnh với ảnh đầu vào. Với hướng tiếp cận ước tính vị trí của máy ảnh, từ cặp ảnh truy vấn và ảnh tham khảo đầu vào, mô hình sẽ tính toán vị trí và hướng quay của máy ảnh trong không gian.  

Nhận thấy rằng hai hướng tiếp cận của bài toán khi đứng riêng đều đã có thể cung cấp những kết quả khả quan trong phạm vi của mình, và kết quả đầu ra cũng như dữ liệu đầu vào của hai hướng tiếp cận tương thích với nhau, nhóm quyết định sẽ kết hợp hai bài toán này thành một quy trình hoàn chỉnh, nhằm bổ trợ cho những khiếm khuyết của nhau. Cụ thể hơn:

\begin{itemize}
    \item Với hướng tiếp cận nhận dạng địa điểm trực quan, nhóm sẽ sử dụng mô hình MixVPR, được xây dựng dựa trên mạng MLP-Mixer. Đây sẽ là nửa đầu của quy trình, cung cấp khả năng mở rộng lên những không gian rộng lớn như thành phố.
    \item Với hướng tiếp cận ước tính vị trí máy ảnh, nhóm sẽ sử dụng hướng tiếp cận 2D-2D được đề xuất trong Map-free Relocalization. Đây sẽ là nửa còn lại của quy trình, cung cấp khả năng đưa ra một dự đoán cụ thể cho quy trình.
\end{itemize}

\end{abstract}