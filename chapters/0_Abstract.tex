\begin{abstract}

    Khả năng định vị toàn cầu đóng vai trò cốt lõi trong những lĩnh vực phụ thuộc vào việc nhận biết và tương tác với môi trường xung quanh như xe tự hành, robot và công nghệ thực tế ảo AR. Trước đây, những công nghệ này sẽ phụ thuộc vào những hệ thống định vị toàn cầu như GPS. Tuy nhiên, công nghệ này vẫn còn những giới hạn nhất định về độ chính xác vị trí cũng như không xác định được hướng xoay. Để đáp ứng nhu cầu ngày càng tăng về độ chuẩn xác, bài toán định vị trực quan - Visual Localization - đã ra đời với mục tiêu thông qua việc thu thập dữ liệu trực quan tại một khu vực nhất định, chúng ta có thể định vị chính xác vị trí cũng như hướng nhìn của một người tại khu vực đó.

    Trong các các hướng tiếp cận hiện đại, nhận thấy các mô hình định vị trực quan cổ điển và định vị trực quan bằng phương pháp hồi quy thường đề xuất các mô hình có tác động tài nguyên huấn luyện lớn và khó phát triển về mặt phạm vi sử dụng, chúng tôi chia bài toán định vị trực quan thành một quy trình hai bước - nhận diện địa điểm trực quan (Visual Place Recognition) và ước tính vị trí của máy ảnh tương đối (Relative Pose Estimation). Trong quá trị nhận dạng địa điểm trực quan, mô hình sẽ nhận vào một ảnh truy vấn và chọn ra một hay nhiều ảnh từ tập ảnh được cung cấp, thể hiện cùng một cảnh với ảnh đầu vào. Từ cặp ảnh truy vấn và ảnh tham khảo đầu vào, mô hình sẽ tính toán vị trí và hướng quay của máy ảnh trong không gian.

    Nhận thấy rằng mỗi bước của bài toán khi hoạt động riêng lẻ đều có những điểm mạnh cũng như những hạn chế riêng, chúng tôi đặt ra mục tiêu nghiên cứu cấu trúc của mô hình và đề xuất phương hướng phát triển để kết hợp hai bước xử lý thành một quy trình hoàn chỉnh, nhằm bổ trợ cho những khiếm khuyết của mỗi bước, với mục đích cuối cùng là xây dựng một mô hình định vị trực quan hoạt động trên cả phạm vi nhỏ lẫn phạm vi rộng như thành thị. Kiến trúc của chúng tôi sử dụng hai mô hình:

    \begin{itemize}
        \item MixVPR \cite{alibey2023mixvpr}, được phát triển bởi Ali-bey và những cộng sự vào năm 2023, cho bước nhận diện địa điểm trực quan, hỗ trợ đưa mô hình định vị lên phạm vị không gian rộng lớn.
        \item Mô hình ước tính vị trí máy ảnh 2D-2D được đề xuất trong Map-free Relocalization \cite{arnold2022mapfree}, phát triển bởi Arnold và những cộng sự vào năm 2022, hỗ trợ tính toán vị trí và hướng quay chính xác của máy ảnh trong không gian từ mỗi cặp ảnh.
    \end{itemize}

\end{abstract}