\begin{abstract}
    Khả năng định vị toàn cầu đóng vai trò cốt lõi trong những lĩnh vực phụ thuộc vào việc nhận biết và tương tác với môi trường xung quanh như xe tự hành, robot và công nghệ thực tế ảo AR. Trước đây, những công nghệ này phụ thuộc vào những hệ thống định vị toàn cầu bằng vệ tinh như GPS. Tuy nhiên, công nghệ này vẫn còn những giới hạn nhất định về độ chính xác cũng như không cung cấp đầy đủ thông tin cần thiết. Để đáp ứng nhu cầu ngày càng tăng về độ chuẩn xác, bài toán định vị trực quan - Visual Localization - đã được phát triển với mục tiêu thông qua việc thu thập dữ liệu trực quan tại một khu vực nhất định, chúng ta có thể định vị chính xác vị trí cũng như hướng nhìn của ảnh chụp tại khu vực đó.

    Sau quá trình khảo sát và nghiên cứu về các hướng tiếp cận bài toán định vị trực quan, chúng tôi nhận thấy hướng tiếp cận truy xuất ảnh, hay còn gọi là nhận diện địa điểm trực quan (Visual Place Recognition) mang nhiều tiềm năng khi có thể hoạt động trên phạm vi rộng mà không tiêu tốn nhiều tài nguyên tính toán. Đổi lại, nhóm phương pháp này thiếu đi khả năng tính toán tư thế (vị trí, góc quay) chính xác của ảnh đầu vào. Ngược lại, các mô hình thuộc nhóm phương pháp ước tính vị trí của máy ảnh tương đối (Relative Pose Regression) tuy có thể tính toán chính xác tư thế tuyệt đối của ảnh, lại chưa được ứng dụng lên trên những khu vực có phạm vi lớn.

    Nhận thấy rằng mỗi bài toán khi hoạt động riêng lẻ đều có những điểm mạnh cũng như những hạn chế riêng, chúng tôi đặt ra mục tiêu nghiên cứu một pipeline có thể kết hợp hai bài toán trên (Visual Place Recognition và Relative Pose Estimation) thành một quy trình hoàn chỉnh, nhằm bổ trợ cho những khiếm khuyết của mỗi bên, với mục đích cuối cùng là xây dựng một mô hình định vị trực quan có thể hoạt động tốt trên nhiều khu vực có phạm vi đi từ một khu vực cụ thể lên đến một thành phố lớn. Cụ thể hơn, pipeline kết hợp của chúng tôi bao gồm hai bộ phận như sau:
    \begin{itemize}
        \item Mô hình MixVPR \cite{alibey2023mixvpr}, được phát triển bởi Ali-bey và nnk vào năm 2023, cho bước nhận diện địa điểm trực quan, hỗ trợ đưa khả năng định vị của mô hình lên phạm vị không gian rộng lớn.
        \item Mô hình ước tính vị trí máy ảnh 2D-2D được đề xuất trong Map-free Relocalization \cite{arnold2022mapfree}, phát triển bởi Arnold và những nnk vào năm 2022, hỗ trợ tính toán vị trí và hướng quay chính xác của ảnh truy vấn trong không gian, sử dụng thông tin trực quan từ cặp ảnh tham khảo và truy vấn.
    \end{itemize}
    Ngoài ra, với mô hình MixVPR \cite{alibey2023mixvpr} thuộc tác vụ Visual Place Recognition, chúng tôi đề xuất một chiến lược khai phá mới nhằm mục tiêu cố gắng nâng cao, cải thiện hiệu suất của tác vụ này trong mô hình kết hợp. Chúng tôi đồng thời cũng sử dụng một tiêu chí đánh giá về độ đáng tin cậy của dự đoán để xác định và loại bỏ được những kết quả sai. Chi tiết sẽ được trình bày ở \textbf{Phương pháp đề xuất}.
\end{abstract}