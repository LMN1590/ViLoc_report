\begin{abstract}

Định vị toàn cầu chính xác (Accurate Global Positioning) luôn là ưu tiên hàng đầu trong các công nghệ phụ thuộc nhiều vào hiểu biết, tính toán sâu về môi trường như xe tự lái, robot tự động và hệ thống định vị. Tuy nhiên, hệ thống định vị toàn cầu (GPS) truyền thống hiện đang đối mặt với một số hạn chế lớn, bao gồm tính không chính xác trong môi trường trong nhà, hạn chế về nhận thức định hướng, khả năng chống chặn tín hiệu và nhiễu đa đường giảm sút.

Bản địa hóa trực quan (Visual Localization) nhằm cung cấp dịch vụ xác định vị trí chính xác hơn bằng việc phân tích thông tin hình ảnh từ môi trường xung quanh. Thay vì dựa vào các phương pháp truyền thống, bản địa hóa trực quan sử dụng hình ảnh để xác định vị trí và hướng của máy ảnh một cách chính xác, mở ra tiềm năng lớn trong việc cải thiện đáng kể khả năng định vị vị trí.

Công nghệ này mang tính cốt lõi đối với các ứng dụng trong các lĩnh vực học máy như thực tế tăng cường, robot và xe tự động lái. Không như Bản địa hóa và lập bản đồ đồng thời (Simultaneous Localization and Mapping), Bản địa hóa trực quan giả định rằng của một cách biểu diễn nào đó của môi trường xung quanh đã được biết trước và không yêu cầu tái tạo môi trường 3D.

Nhận thấy rằng hướng tiếp cận học sâu (Deep Learning) đã và đang mang lại những kết quả đầy hứa hẹn trong cải thiện độ chính xác cũng như độ chắc chắn của Bản địa hóa trực quan, nhóm chúng tôi quyết định sẽ áp dụng các chiến thuật học sâu vào công cuộc cải thiện các phương pháp Bản địa hóa trực quan hiện có:

\begin{itemize}
    \item Với hướng tiếp cận kiến trúc, chúng tôi sẽ thay thế những cơ chế thành phần hiện tại của các mô hình bằng những mô hình học sâu.
    \item Với hướng tiếp cận end-to-end, chúng tôi cải tiến quá trình huấn luyện bằng những cách như áp dụng nhiều loại dữ liệu khác, xây dựng kiến trúc hiện đại hơn...

\end{itemize}

\end{abstract}