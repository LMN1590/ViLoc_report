\chapter{THẢO LUẬN}
\section{Sự liên kết giữa mô hình truy xuất ảnh và mô hình hồi quy tương đối}
Mặc dù có nhiều cách để thực hiện việc hồi quy ra vị trí ảnh:
\begin{itemize}
    \item Hồi quy dựa trên bản đổ đám mây điểm 3D toàn cục
    \item Hồi quy dựa trên bản đồ đám mây điểm 3D cục bộ
    \item Hồi quy dựa trên vị trí của các ảnh được truy xuất
\end{itemize}
Đa số các mô hình đều sử dụng cùng một cách biểu diễn khu vực, dựa trên những đặc trưng đã được thu gọn trên hình. Những đặc trưng này thường sẽ được huấn luyện cho những bài toán truy xuất địa danh, nhận diện khu vực. Vậy nên các đặc trưng này sẽ có những giá trị giống nhau khi các ảnh cùng nhìn một tòa nhà, một địa danh, kể cả dưới những góc nhìn khác nhau. Vậy nên, có thể kết quả của một mô hình thực hiện cả 2 tác vụ rời rạc này sẽ không đạt được kết quả tối ưu.\cite{pion2020benchmarking}

\textbf{Lưu ý:} Liệt kê ra những yêu cầu về đặc trưng cho những phương pháp khác nhau ở trên, cũng trong bài \cite{pion2020benchmarking}
\section{Tìm hiểu giới hạn của mô hình hồi quy vị trí tuyệt đối dựa trên mạng Nơ-ron tích chập}
Các mô hình hồi quy vị trí tuyệt đối sẽ xây dựng cách biểu diễn của khu vực trong tập dữ liệu một cách bao hàm. Thông qua đó, một số địa điểm bên trong khu vực được biểu diễn bởi tập dữ liệu sẽ được ngầm chọn làm điểm mốc. Điều này dẫn đến việc khi được huấn luyện trên những tập dữ liệu có hạn chế về mặt không gian thì mô hình hồi quy vị trí tuyệt đối sẽ không thể nào tổng quát hóa ra những khu vực không có trong tập dữ liệu ban đầu được mà phải huấn luyện lại trên tập dữ liệu của khu vực đó.\cite{sattler2019understanding}

Điều tương tự cũng có thể xảy ra với hồi quy vị trí tương đối, khi mà những ảnh truy xuất được có vị trí không đủ tổng quát thì có thể ảnh hưởng xấu đến kết quả của quá trình hồi quy vị trí tương đối

\textbf{Lưu ý}: Cần ghi ra phương pháp, cấu trúc mô hình được sử dụng trong bài. Cấu trúc khá đơn giản nên có thể không tổng quát hóa được. Cần ghi thêm các thí nghiệm của người ta.

\section{Tìm hiểu giới hạn của một số tập dữ liệu}
Một số tập dữ liệu quá thưa, không phản ánh được chính xác vùng mà tập dữ liệu muốn miêu tả, mà nó dày quá thì mô hình chạy không nổi\cite{berton2022rethinking} 