\chapter{ĐO ĐẠC VÀ ĐÁNH GIÁ}

\section{Một số tập dữ liệu phổ biến được sử dụng}
\subsection{Tập dữ liệu phạm vi vừa - nhỏ}
\subsubsection*{7Scenes}
Tập dữ liệu 7-Scenes \cite{6619221} bao gồm các ảnh RGB-D thuộc bảy khung cảnh khác nhau được chụp từ một máy ảnh cầm tay Kinect RGB-D ở độ phân giải 640x480. Bảy khung cảnh bap gồm: "Chess", "Fire", "Heads", "Office", "Pumpkin", "RedKitchen" và "Stairs". Với mỗi cảnh có vài chuỗi khung ảnh RGB-D. Mỗi chuỗi bao gồm khoảng từ 1000 đến 5000 khung ảnh. Mỗi khung gồm: ảnh màu, độ sâu và vị trí.
\begin{figure}[H]
    \centering
    \includegraphics[width=\textwidth]{pics/Chapter2/7scenes.png}
    \caption{Minh họa tập dữ liệu 7-Scenes \cite{6619221}}
\end{figure}
\subsubsection*{Cambridge Landmark}
Tập dữ liệu Cambridge Landmarks \cite{kendall2016posenet} là một tập dữ liệu định vị thành thị bao gồm năm khung cảnh khác nhau. Các yếu tố dày đặc quan trọng như phương tiện giao thông hay người đi bộ cũng xuất hiện trong tập dữ liệu này, ngoài ra dữ liệu cũng được thu thập ở nhiều thời điểm trong ngày đại diện cho các yếu tố ánh sáng và điều kiện thời tiết khác nhau. Cambridge Landmarks được tạo ra nhờ vào việc áp dụng các kỹ thuật tái tạo kiến trúc từ chuyển động. Một chiếc điện thoại thông minh Google LG Nexus 5 được một người đi bộ trên phố sử dụng để ghi lại đoạn phim chất lượng cao cho mỗi cảnh. Mỗi đoạn phim sau đó được lấy mẫu với tần số 2Hz để trích xuất ảnh cho quy trình tái tạo kiến trúc từ chuyển động. Mỗi vị trí máy ảnh cách nhau khoảng 1m. Đây là một trong những tập dữ liệu phổ biến nhất trong việc huấn luyện và đo đạc kết quả trên các tác vụ RPR.
\begin{figure}[H]
    \centering
    \includegraphics[width=\textwidth]{pics/Chapter2/cambridge.png}
    \caption{Minh họa tập dữ liệu Cambridge Landmarks \cite{kendall2016posenet}}
\end{figure}
\subsubsection*{Niantic Map-free Relocalization Dataset}
Tập dữ liệu Niantic Map-free Relocalization \cite{arnold2022mapfree} là một tập dữ liệu được thu thập chủ yếu để giúp ích cho phương pháp định vị Map-free \cite{arnold2022mapfree}. Tập dữ liệu bao gồm 655 cảnh bên ngoài với mỗi cảnh chứa một "địa điểm đáng chú ý" như một pho tượng, cổng, bảng hiệu,... sao cho địa điểm đó phải được xác định rõ trong một bức ảnh. Các cảnh được chia ra thành 460 cảnh phục vụ cho tác vụ huấn luyện, 65 cảnh phục vụ cho tác vụ kiểm tra quy trình huấn luyện và 130 cảnh phục vụ cho quá trình kiểm thử. Mỗi ảnh trong tập huấn luyện đều được gắn kèm vị trí tuyệt đối. Với tập kiểm thử và kiểm tra quy trình, mỗi cảnh được kèm theo một ảnh đại diện cũng như vị trí tuyệt đối tại cảnh. Ngoài ra, ma trận tham số nội tại của máy ảnh cũng được gắn kèm theo mỗi ảnh trong tập dữ liệu.
\begin{figure}[H]
    \centering
    \includegraphics[width=\textwidth]{pics/Chapter2/niantic.png}
    \caption{Minh họa tập dữ liệu Niantic Map-free Relocalization \cite{arnold2022mapfree}}
\end{figure}
\subsection{Tập dữ liệu thành thị phạm vi rộng}
%\subsubsection*{Aachen Day-Night}
%Tập dữ liệu Aachen Day-Night \cite{Sattler2012ImageRF} bao gồm 14.607 ảnh được chụp với nhiều máy ảnh khác nhau bao phủ cả thành phố Aachen thuộc quốc gia Đức. Các ảnh dữ liệu được chụp ở nhiểu thời điểm trong ngày và trong năm, cụ thể là khoảng thời gian trong 2 năm. Hệ quả mang lại là tập dữ liệu bao phủ nhiều điều kiện ngoại cảnh như thời tiết, ánh sáng cũng như sự thay đổi của công trình kiến trúc trong khu vực.
%\begin{figure}[H]
%	\centering
%	\includegraphics[width=\textwidth]{pics/Chapter2/aachen.png}
%	\caption{Minh họa tập dữ liệu Aachen Day-Night \cite{Sattler2012ImageRF}}
%\end{figure}
\subsubsection*{Pittsburgh 250k \cite{6618963}}
Tập dữ liệu Pittsburgh 250k \cite{6618963} là một tập dữ liệu tương đối rộng bao phủ thành phố Pittsburgh, Mỹ. Đây là một tập dữ liệu tương đối phổ biến trong thị giác máy tính, cụ thể là ở tác vụ nhận điện địa điểm trực quan, truy xuất ảnh và định vị trực quan. Tập dữ liệu được đánh nhãn về vị trí chụp cho mỗi ảnh, nhưng lại thiếu thông tin về góc quay của máy ảnh.

\subsubsection*{GSV-Cities \cite{Ali_bey_2022}}
Tập dữ liệu GSV-Cities \cite{Ali_bey_2022} bao phủ một vùng địa lý cực kỳ rộng lớn với hơn 40 thành phố xuyên lục địa trong khoảng thời gian 14 năm liên tục. GSV-Cities chứa khoảng hơn 530.000 ảnh - khoảng hơn 62.000 vị trí. Mỗi vị trí có khoảng từ 4 đến 20 ảnh. Đồng thời mỗi vị trí cách nhau một khoảng ít nhất 100m.
\begin{figure}[H]
    \centering
    \includegraphics[width=\textwidth]{pics/Chapter2/gsv.png}
    \caption{Minh họa tập dữ liệu GSV-Cities \cite{Ali_bey_2022}}
\end{figure}

%\subsubsection*{SF-XL}
%Tập dữ liệu San Francisco Extra Large (SF-XL) \cite{berton2022rethinking} được tạo nên từ 3.43 triệu ảnh 360 độ thu thập từ kho ảnh Google Streetview. Các ảnh này sau đó được cắt ra thành 41.2 triệu ảnh. Mỗi ảnh cắt ra đều được gắn nhãn 6DoF (bao gồm cả GPS). Dữ liệu được thu thập từ năm 2009 đến năm 2021, dẫn đến việc tập dữ liệu bao phủ nhiều điều kiện ngoại cảnh như thời tiết, ánh sáng cũng như sự thay đổi của công trình kiến trúc trong khu vực.
%\begin{figure}[H]
%	\centering
%	\includegraphics[width=\textwidth]{pics/Chapter2/sfxl.png}
%	\caption{Minh họa tập dữ liệu San Francisco Extra Large \cite{berton2022rethinking}}
%\end{figure}

\subsection{Kết luận}
Tập dữ liệu GSV-Cities \cite{Ali_bey_2022} được sử dụng nhằm tinh chỉnh mô hình kết hợp của chúng tôi. Sau đó, các tập dữ liệu Cambridge Landmark \cite{kendall2016posenet} và Pittsburgh250k \cite{6618963} được sử dụng trong quá trình thí nghiệm và đo đạc kết quả của mô hình kết hợp.

\section{Kiểm nghiệm}
\subsection{Mô hình MixVPR}
\subsubsection*{Mô tả quá trình thí nghiệm}

Để kiểm chứng kết quả đã được công bố trên bài báo khoa học của nhóm nghiên cứu tác giả, chúng tôi đã tiến hành chạy mô hình MixVPR \cite{alibey2023mixvpr} trên tập dữ liệu Pittsburgh 250k \cite{6618963} và tập dữ liệu Pittsburgh 30k \cite{6618963}.

Các thang đo kết quả được sử dụng trong báo cáo này là recall@k, thể hiện tỷ lệ của truy xuất thành công trên tổng số lượng truy xuất và một truy xuất hình được xem là thành công khi ảnh được truy xuất nằm trong vòng 25m xung quanh ảnh truy vấn. Để thích hợp cho việc chạy trên thiết bị cá nhân với cấu hình yếu hơn so với thiết bị của tác giả, chúng tôi đã hiệu chỉnh cài đặt kích thước mỗi batch chuyển từ 120 xuống 10, các cài đặt khác được giữ nguyên so với bài nghiên cứu.

\subsubsection*{Kết quả thí nghiệm}

\begin{table}[H]
    \centering
    \begin{tabular}{|l|l|l|l|l|l|l|l|}
        \hline
                       & \textbf{R@1} & \textbf{R@5} & \textbf{R@10} & \textbf{R@15} & \textbf{R@20} & \textbf{R@50} & \textbf{R@100} \\ \hline
        Pittsburgh30k  & 91.64        & 95.55        & 96.35         & 96.99         & 97.34         & 98.25         & 98.69          \\ \hline
        Pittsburgh250k & 94.32        & 98.22        & 98.84         & 99.13         & 99.34         & 99.54         & 99.66          \\ \hline
    \end{tabular}
\end{table}

\subsubsection*{Nhận xét}

Các số liệu đo đạc thu được từ thí nghiệm của chúng tôi là hoàn toàn giống với bảng kết quả đã được công bố của nhóm tác giả với recall@1 đạt 94.32\%, recall@5 đạt 98.22\%, recall@10 đạt 98.84\%. Ngoài ra, chúng tôi nhận thấy rằng các cài đặt được thay đổi không ảnh hưởng đến kết quả đầu ra của mô hình mà thay vào đó chỉ ảnh hưởng đến tài nguyên thiết bị cũng như tốc độ thực thi của mô hình. Cuối cùng, với tác vụ nhận diện địa điểm trực quan, chúng tôi cho rằng các chỉ số kết quả này là đủ tốt để tiến hành sử dụng MixVPR trong mô hình kết hợp.

\subsection{Mô hình Map-free Relocalization}
\subsubsection*{Mô tả quá trình thí nghiệm}

Để kiểm chứng kết quả đã được công bố trên trang chủ của nhóm nghiên cứu tác giả công trình, chúng tôi đã tiến hành chạy quá trình hồi quy tư thế tương quan 2D - 2D của mô hình Niantic Map-free Relocalization trên tập dữ liệu kiểm thử do chính tác giả cung cấp. Cụ thể tập dữ liệu bao gồm 15.000 ảnh chia thành 130 cảnh khác nhau với mỗi cảnh bao gồm một vật thể chú ý đặt ở tâm như mô tả bài báo.

Các thang đo kết quả được sử dụng trong báo cáo này bao gồm độ lệch vị trí (m), độ lệch góc quay (độ) và sai số phản chiếu điểm 3D ảo VCRE (điểm ảnh). Các cài đặt của mô hình được giữ nguyên như trong bài nghiên cứu của tác giả.

\subsubsection*{Kết quả thí nghiệm}

\begin{table}[H]
    \centering
    \begin{tabular}{|l|c|c|c|c|}
        \hline
        Method                                                                                                       & \multicolumn{1}{l|}{\begin{tabular}[c]{@{}l@{}}Precision \\ (Err \textless 25cm, 5°)\end{tabular}} & \multicolumn{1}{l|}{\begin{tabular}[c]{@{}l@{}}Median Trans. \\ Error (m)\end{tabular}} & \multicolumn{1}{l|}{\begin{tabular}[c]{@{}l@{}}Median Rot. \\ Error (°)\end{tabular}} & \multicolumn{1}{l|}{\begin{tabular}[c]{@{}l@{}}Median Reproj. \\ Error (px)\end{tabular}} \\ \hline
        \begin{tabular}[c]{@{}l@{}}DPT-KITTI \& \\ SuperGlue \\ (Ess.Mat. + D.Scale) \\ (Author)\end{tabular}        & 15.4\%                                                                                             & 1.98                                                                                    & 30.5                                                                                  & 167.6                                                                                     \\ \hline
        \textbf{\begin{tabular}[c]{@{}l@{}}DPT-KITTI \& \\ SuperGlue \\ (Ess.Mat. + D.Scale) \\ (Ours)\end{tabular}} & 15.6\%                                                                                             & 1.92                                                                                    & 26.1                                                                                  & 161.1                                                                                     \\ \hline
    \end{tabular}
    \caption{Bảng so sánh kết quả công bố và kết quả kiểm thử mô hình 2D - 2D Map-free}
\end{table}

\subsubsection*{Nhận xét}

Các số liệu đo đạc thu được từ thí nghiệm của chúng tôi tương đối sát với kết quả do nhóm tác giả nghiên cứu đã công bố trên trang chủ với độ lệch vị trí khoảng gần 2m và độ lệch góc quay khoảng 26 độ.

\section{Thí nghiệm trên pipeline kết hợp}
\subsection*{Mô tả quá trình thí nghiệm}

Chúng tôi tiến hành tinh chỉnh mô hình MixVPR \cite{alibey2023mixvpr} với chiến lược khai phá mới mà chúng tôi đã đề xuất trên cùng tập dữ liệu GSV-Cities \cite{Ali_bey_2022}. Ở tập dữ liệu GSV-Cities, dữ liệu ảnh được sắp xếp theo từng khu vực trong từng thành phố trên khắp các lục địa. Bài báo gốc đã xem quá trình huấn luyện như một bài toán phân loại ảnh theo khu vực, sử dụng Multi-Similarity Miner và Loss \cite{wang2019multi} dựa vào mã định danh của khu vực. Trong khi đó, Frustum Miner của chúng tôi phân loại các ảnh trong một khu vực theo độ trùng lắp về khung hình giữa các ảnh nhằm mục đích cố gắng giúp tác vụ RPR hoạt động hiệu quả và dễ dàng hơn.

Chúng tôi đã tiến hành chọn lọc một số khu vực đáp ứng đủ điều kiện mật độ ảnh mà nhóm đặt ra để đảm bảo rằng mỗi khu vực đều xuất hiện tính đa dạng về góc nhìn. Kích thước batch được chọn là 8 khu vực với mỗi nơi gồm 16 ảnh, tức một batch chứa 128 ảnh. Một lớp Feature Mixer được đóng băng trong quá trình finetune. Chúng tôi tiến hành finetune qua 5 epoch với learning rate được chỉnh về 0.0005. Các ngưỡng frustum và góc quay được đặt lần lượt là 1.25/2 và 60 độ theo thứ tự. Các cài đặt không được nêu lên được giữ nguyên như bài báo gốc.

Chúng tôi tiến hành kiểm thử kết quả mô hình kết hợp trên các tập dữ liệu thường dùng cho hai tác vụ VPR và RPR riêng biệt: trên tập dữ liệu thành thị phạm vi rộng nhưng mật độ ảnh thưa (Pitts250k-test \cite{6618963}) và trên một tập dữ liệu kích thước vừa nhưng mật độ ảnh cao (Cambridge Landmark \cite{kendall2016posenet}). Mô hình kết hợp của chúng tôi sử dụng ngưỡng tương quan tốt nhất tương ứng với các tập dữ liệu (inlier > 5 với Pitts250k-test, inlier > 35 với Cambridge Landmark). Kết quả được thể hiện ở bảng \ref{tab:pitts250k_vpr_comparison} và \ref{tab:cambridge_vpr_comparison}.

\subsection*{Tập dữ liệu Pitts250k-test}
Với tác vụ VPR, chúng tôi tiến hành đo đạc kết quả của mô hình kết hợp sau khi tinh chỉnh, sau đó so sánh với các mô hình cơ sở khác đã được huấn luyện sẵn bao gồm: MixVPR \cite{alibey2023mixvpr}, NetVLAD \cite{arandjelovic2016netvlad} và AnyLoc \cite{keetha2023anyloc}. Thí nghiệm được tiến hành trên tập dữ liệu Pittsburgh250k-test \cite{6618963}. Với những mô hình VPR truyền thống, tư thế của ảnh truy xuất top 1 được dùng làm tư thế cuối cùng của ảnh truy vấn. Ngoài ra, do tập dữ liệu Pittsburgh250k không chứa đựng thông tin góc quay cần thiết cho thí nghiệm, thang đo kết quả được sử dụng trong thí nghiệm này là trung bình độ lệch vị trí giữa tư thế đầu ra và tư thế chính xác của ảnh đầu vào với đơn vị (mét). Kết quả được thể hiện ở bảng \ref{tab:pitts250k_vpr_comparison}.

Với NetVLAD \cite{arandjelovic2016netvlad}, chúng tôi sử dụng mô hình đã được huấn luyện sẵn trên tập Pittsburgh250k. Với AnyLoc \cite{keetha2023anyloc}, chúng tôi sử dụng các thiết lập mặc định như trong bài báo khoa học của nhóm nghiên cứu tác giả. Do mô hình RPR yêu cầu các tập dữ liệu ảnh cần phải cung cấp các thông số nội tại, chúng tôi đã sử dụng mô hình COLMAP\cite{schoenberger2016sfm} để tính toán các tham số nội tại cho tập dữ liệu Pittsburgh250k trước khi đưa vào thí nghiệm đo đạc, các tham số này chỉ mang tính chất tham khảo và không nên áp dụng phương pháp này khi xây dựng tập dữ liệu thành thị mới.

\bgroup
\def\arraystretch{1.4}%
\setlength\tabcolsep{10 pt}
\begin{table}
    \centering
	\begin{tabular}{|l|c|c|c|}
	\hline
	                                                        & Dim   & \begin{tabular}[c]{@{}c@{}}Success\\ Rate $\uparrow$\end{tabular} & \begin{tabular}[c]{@{}c@{}}Position \\ (meters) $\downarrow$\end{tabular} \\ \hline
	AnyLoc \cite{keetha2023anyloc}         & 12288 & 1.00                                                                & 42.56                                                                     \\
	MixVPR \cite{alibey2023mixvpr}         & 4096  & 1.00                                                                & {\ul 21.82}                                                               \\
	NetVLAD \cite{arandjelovic2016netvlad} & 32768 & 1.00                                                                & 84.88                                                                     \\ \hline
	Ours                                                    & 4096  & 0.93                                                                & \textbf{10.09}                                                            \\ \hline
	\end{tabular}
    \vspace{10pt}
    \caption[Kết quả định vị trực quan trên tập Pittsburgh250k-test]{Kết quả định vị trực quan trên tập Pittsburgh250k-test \cite{6618963}. Thang đo trung bình độ lệch vị trí được áp dụng cho các mô hình được so sánh. Kết quả tốt nhất và tốt thứ nhì được đánh dấu (với in đậm/gạch dưới, theo thứ tự).}
    \label{tab:pitts250k_vpr_comparison}
\end{table}
\egroup

\subsubsection*{Nhận xét}
Thực hiện tác vụ VPR trên tập dữ liệu Pittsburgh250k-test \cite{6618963}, mô hình kết hợp của chúng tôi đạt kết quả tốt nhất về mặt sai lệch vị trí, nhưng lại đánh đổi tỷ lệ hồi quy tư thế ảnh thành công.

Sự suy giảm về tỷ lệ hồi quy tư thế thành công ở mô hình kết hợp có thể được gây ra bởi những trường hợp các ảnh truy xuất được từ module VPR chưa có đủ độ trùng lặp với ảnh đầu vào để tác vụ RPR có thể hoàn thành tốt việc tính toán tư thế, đi từ việc mô hình cơ sở xử lý tác vụ feature matching hoạt động chưa đủ hiệu quả với các tập dữ liệu thành thị đa dạng về góc quay, nên không thể xác định được những cặp điểm tương quan trên cặp ảnh.

Hiện tượng kiến trúc trùng lặp (các cảnh cách xa nhau nhưng có độ tương đồng cao về mặt hình ảnh, đặc trưng) ở tập dữ liệu Pittsburgh250k \cite{6618963} có thể dẫn đến việc truy xuất ảnh sai và từ đó hồi quy sai tư thế. Ngoài ra, tác vụ dự đoán độ sâu của mô hình cơ sở tồn tại nhiều sai sót tiềm ẩn, dẫn đến lỗi tích lũy trong quy trình hồi quy tư thế của toàn module RPR.

\subsection*{Tập dữ liệu Cambridge Landmark}

Với tác vụ RPR, chúng tôi tiến hành đo đạc kết quả của mô hình kết hợp và so sánh với các mô hình cơ sở khác đã được huấn luyện sẵn bao gồm: EssNet \cite{zhou2020learn}, NC-EssNet \cite{zhou2020learn} và Relformer \cite{idan2023learning}. Đồng thời, chúng tôi cũng áp dụng các mô hình VPR được so sánh ở thí nghiệm trước lên thí nghiệm này nhằm kiểm chứng độ hiệu quả.

Thí nghiệm được tiến hành trên tập dữ liệu Cambridge Landmark \cite{kendall2016posenet}. Thang đo kết quả được sử dụng trong thí nghiệm này là trung bình độ lệch vị trí và góc quay lần lược với đơn vị mét, độ. Do trọng tâm của chúng tôi là việc áp dụng định vị trực quan lên dữ liệu chưa từng thấy, không có mô hình nào trong thí nghiệm này từng được huấn luyện trên tập dữ liệu Cambridge Landmark \cite{kendall2016posenet}. Kết quả được thể hiện ở bảng \ref{tab:cambridge_vpr_comparison}.

\bgroup
\def\arraystretch{1.4}%
\setlength\tabcolsep{10 pt}
\begin{table}
    \centering
    \begin{tabular}{|l|c|c|c|}
        \hline
                                               & \begin{tabular}[c]{@{}c@{}}Success\\ Rate $\uparrow$\end{tabular} & \begin{tabular}[c]{@{}c@{}}Position\\ (meters) $\downarrow$\end{tabular} & \begin{tabular}[c]{@{}c@{}}Orientation \\ (degrees) $\downarrow$\end{tabular} \\ \hline
        EssNet \cite{zhou2020learn}            & 1.00                                                                & 10.40                                                                    & 85.80                                                                         \\
        NC-EssNet \cite{zhou2020learn}         & 1.00                                                                & 7.98                                                                     & 24.40                                                                         \\
        Relformer \cite{idan2023learning}      & 1.00                                                                & {\ul 3.35}                                                               & 10.60                                                                         \\ \hline
        NetVLAD \cite{arandjelovic2016netvlad} & 1.00                                                                & 3.99                                                                     & 10.03                                                                         \\
        AnyLoc \cite{keetha2023anyloc}         & 1.00                                                                & 4.46                                                                     & 10.42                                                                         \\
        MixVPR \cite{alibey2023mixvpr}         & 1.00                                                                & 4.04                                                                     & {\ul 7.44}                                                                    \\ \hline
        Ours                                   & 0.95                                                                & \textbf{3.16}                                                            & \textbf{4.35}                                                                 \\ \hline
    \end{tabular}
    \vspace{10pt}
    \caption[Kết quả định vị trực quan trên tập Cambridge Landmark]{Kết quả định vị trực quan trên tập Cambridge Landmark \cite{kendall2016posenet}. Thang đo trung bình độ lệch vị trí và góc quay được áp dụng cho các mô hình được so sánh. Kết quả tốt nhất và tốt thứ nhì được đánh dấu (với in đậm/gạch dưới, theo thứ tự).}
    \label{tab:cambridge_vpr_comparison}
\end{table}
\egroup

\subsubsection*{Nhận xét}
Thực hiện tác vụ RPR trên tập dữ liệu Cambridge Landmark \cite{kendall2016posenet}, mô hình kết hợp của chúng tôi đạt kết quả tốt nhất về sai số vị trí và góc quay nhưng giảm nhẹ tỷ lệ hồi quy tư thế ảnh thành công. Chúng tôi kết hợp độ khái quát hóa tốt của tác vụ VPR và khả năng tính toán tư thế tuyệt đối chuẩn xác của tác vụ RPR, dẫn đến việc mô hình kết hợp đạt được kết quả tốt nhất dù chưa từng được huấn luyện trên tập dữ liệu này. Tác vụ RPR hoạt động tốt hơn khi mật độ ảnh cùng nhìn vào một địa điểm cao. Cụ thể là mô hình cơ sở feature matching được cải thiện hiệu suất khi các ảnh có độ trùng lặp cao.

Các mô hình VPR tuy có độ chuẩn xác cao trong việc truy xuất ảnh nhưng lại không có khả năng tính toán được tư thế chính xác của ảnh được chụp dẫn đến kết quả có phần tệ hơn. Các mô hình RPR tuy có khả năng tính toán tốt tư thế tuyệt đối trong những tập dữ liệu vừa và nhỏ song lại có thể mắc phải tình trạng quá khớp vào tập dữ liệu huấn luyện, dẫn đến kết quả không quá khả quan trên các tập dữ liệu chưa từng thấy.

\section{Ablation Study}
Chúng tôi tiến hành nghiên cứu, đo đạc và ghi lại kết quả của việc thử nghiệm các hướng cải thiện khác nhau với mô hình kết hợp. Kết quả được ghi nhận tại các bảng \ref{tab:inliers}, \ref{tab:vpr_rpr}, \ref{tab:reranking}, \ref{tab:weighted_pose} và \ref{tab:best_overlap}.

% Ablation study inliers
\subsection{Ngưỡng cặp điểm tương quan (inlier correspondence)}

Chúng tôi tiến hành thực hiện những thí nghiệm nhằm nghiên cứu ảnh hưởng cũng như những đánh đổi về mặt kết quả của mô hình kết hợp ở các ngưỡng tự tin khác nhau trên hai tập dữ liệu Pittsburgh250k-test \cite{6618963} và Cambridge Landmark \cite{kendall2016posenet}. Kết quả được thể hiện ở bảng \ref{tab:inliers}. Qua quá trình thí nghiệm, chúng tôi nhận thấy khi ngưỡng inlier tăng cao, độ lệch vị trí và độ lệch góc quay được cải thiện đáng kể. Đổi lại, tỷ lệ hồi quy tư thế ảnh thành công bị suy giảm nhẹ. Khi khảo sát trên tập dữ liệu phạm vi rộng, có mật độ ảnh thưa và tồn tại những cụm ảnh có tư thế gần nhau nhưng góc nhìn lại có độ lệch lớn, việc áp dụng những ngưỡng nhỏ như $inlier > 5$ và $inlier > 2$ đã đem lại ảnh hưởng tích cực lên sai số mô hình. Điều này thể hiện rằng số lượng cặp điểm tương quan giữa những cặp ảnh là rất hiếm và có ảnh hưởng lớn đến quá trình định vị của pipeline. Ngược lại, vì tập dữ liệu Cambridge Landmark chứa nhiều góc quay cùng hướng vào một tâm điểm cụ thể (tòa nhà, ngã tư, \dots), dẫn đến việc số lượng cặp điểm tương quan trung bình giữa những cặp ảnh tương đối cao. Vì vậy, thí nghiệm không thể hiện ảnh hưởng quá nhiều lên sai số mô hình. Thêm vào đó, với những tập dữ liệu khác nhau, ảnh hưởng tích cực lên sai lệch về vị trí và góc quay đạt cực đại tại một ngưỡng tương quan khác nhau. Hướng tiếp cận này thể hiện nhiều tiềm năng phát triển, đặc biệt vào việc nghiên cứu các giải thuật phù hợp để xác định ngưỡng tối ưu cho các cơ sở dữ liệu định vị trực quan.

% table
\bgroup
\def\arraystretch{1.4}%
\setlength\tabcolsep{10 pt}
\begin{table}[H]
    \centering
    \begin{tabular}{|c|c|c|c|c|}
        \hline
        \multicolumn{1}{|l|}{}                                                                                            & \begin{tabular}[c]{@{}c@{}}Inliers\\ Threshold\end{tabular} & \begin{tabular}[c]{@{}c@{}}Success\\ Rate $\uparrow$\end{tabular} & \begin{tabular}[c]{@{}c@{}}Position\\ (meters) $\downarrow$\end{tabular} & \begin{tabular}[c]{@{}c@{}}Orientation\\ (degrees) $\downarrow$\end{tabular} \\ \hline
        \multirow{5}{*}{\textbf{\begin{tabular}[c]{@{}c@{}}Pittsburgh250k\\ -test \cite{6618963}\end{tabular}}}           & Inliers \textgreater 0                                      & 0.98                                                                & 18.32                                                                    & -                                                                        \\
                                                                                                                          & Inliers \textgreater 2                                      & 0.97                                                                & 13.44                                                                    & - \\
                                                                                                                          & Inliers \textgreater 5                                      & 0.93                                                                & 10.09                                                                    & -                                                                        \\
                                                                                                                          & Inliers \textgreater 10                                     & 0.87                                                                & 9.02                                                                     & -                                                                        \\
                                                                                                                          & Inliers \textgreater 25                                     & 0.66                                                                & 8.35                                                                     & - \\ \hline
        \multirow{4}{*}{\textbf{\begin{tabular}[c]{@{}c@{}}Cambridge \\ Landmark \cite{kendall2016posenet}\end{tabular}}} & Inliers \textgreater 0                                      & 1.00                                                                & 3.19                                                                     & 4.80                                                                         \\
                                                                                                                          & Inliers \textgreater 25                                     & 0.97                                                                & 3.18                                                                     & 4.65                                                                         \\
                                                                                                                          & Inliers \textgreater 35                                     & 0.95                                                                & 3.16                                                                     & 4.35                                                                         \\
                                                                                                                          & Inliers \textgreater 50                                     & 0.88                                                                & 3.14                                                                     & 4.34                                                                         \\ \hline
    \end{tabular}
    \caption{Ablation Study với các ngưỡng tương quan của mô hình kết hợp trên tập dữ liệu Pittsburgh250k-test \cite{6618963} và Cambridge Landmark \cite{kendall2016posenet}}
    \label{tab:inliers}
\end{table}
\egroup

% Ablation study VPR + RPR
\subsection{Sử dụng các mô hình VPR khác cho mô hình kết hợp}

Nhận thấy tiềm năng của các mô hình VPR AnyLoc \cite{keetha2023anyloc} và MixVPR cơ sở \cite{alibey2023mixvpr} được sử dụng trong quá trình thí nghiệm và so sánh. Chúng tôi tiến hành nghiên cứu độ hiệu quả của các mô hình này khi kết hợp với module RPR trên tập dữ liệu Pittsburgh250k \cite{6618963}. Đồng thời, NetVLAD \cite{arandjelovic2016netvlad}, một mô hình trích xuất và tổng hợp đặc trưng tương đối cơ bản cũng được kết hợp để so sánh với các mô hình VPR hiện đại. Kết quả được thể hiện ở bảng \ref{tab:vpr_rpr}. Có thể thấy các mô hình có tỷ lệ hồi quy tư thế ảnh thành công ngang nhau, với MixVPR dẫn đầu về vị trí, AnyLoc ở vị trí thứ hai và NetVLAD ở vị trí cuối cùng. Vì MixVPR được huấn luyện trên tập dữ liệu GSV-Cities \cite{Ali_bey_2022} trong khi AnyLoc sử dụng các mô hình cơ sở tiền huấn luyện, chúng tôi cho rằng việc áp dụng các mô hình cơ sở tiền huấn luyện lên tác vụ VPR dù có độ tổng quát hóa cao vẫn chưa đạt được kết quả lý tưởng nhất trong miền dữ liệu thành thị.

% table
\bgroup
\def\arraystretch{1.4}%
\setlength\tabcolsep{10 pt}
\begin{table}[H]
    \centering
    \begin{tabular}{|l|c|c|c|c|}
        \hline
                      & Dim   & \begin{tabular}[c]{@{}c@{}}Success\\ Rate $\uparrow$\end{tabular} & \begin{tabular}[c]{@{}c@{}}Position\\ (meters) $\downarrow$\end{tabular}  \\ \hline
        AnyLoc + RPR  & 12288 & 0.99                                                                & 41.39                                                                                                                                          \\
        MixVPR + RPR  & 4096  & 0.98                                                                & 18.96                                                                                                                                          \\
        NetVLAD + RPR & 32768 & 0.99                                                                & 77.39                                                                                                                                          \\ \hline
    \end{tabular}
    \caption{Ablation Study kết quả các baseline khi kết hợp với module RPR trên tập dữ liệu Pittsburgh250k-test \cite{6618963}}
    \label{tab:vpr_rpr}
\end{table}
\egroup

% Ablation study Reranking
\subsection{Kỹ thuật tái xếp hạng (reranking) trên mô hình kết hợp}

Chúng tôi tiến hành kiểm thử hiệu quả của hướng tiếp cận tái xếp hạng lên mô hình kết hợp trên tập dữ liệu Pittsburgh250k-test \cite{6618963}. Chúng tôi chọn các ngưỡng (k=1, 3, 5) ảnh để tính toán và so sánh. Kết quả được thể hiện ở bảng \ref{tab:reranking}. Do tính đa dạng về góc quay của tập dữ liệu, việc thực hiện tái xếp hạng lên danh sách ảnh đã truy vấn được không cải thiện được hiệu quả mô hình. Ngược lại, quá trình này có thể làm tăng nhẹ sai số mô hình. Chúng tôi cho rằng vấn đề trên có thể đã được tạo ra bởi hai lý do chính: không có nhiều khác biệt giữa hiệu quả của hai lần truy xuất ảnh, trường hợp danh sách ảnh truy xuất được chứa nhiều ảnh không chính xác hơn ảnh chính xác (ngoài các ảnh đầu tiên được truy xuất là chính xác thì các ảnh sau chỉ làm cho kết quả bị sai lệch chứ không cải thiện).

% table
\bgroup
\def\arraystretch{1.4}%
\setlength\tabcolsep{10 pt}
\begin{table}[H]
    \centering
    \begin{tabular}{|l|c|c|c|}
        \hline
                        & \begin{tabular}[c]{@{}c@{}}Success\\ Rate $\uparrow$\end{tabular} & \begin{tabular}[c]{@{}c@{}}Position\\ (meters) $\downarrow$\end{tabular} \\ \hline
        Reranking (k=1) & 0.9886                                                              & 18.32                                                                                                                                            \\
        Reranking (k=3) & 0.9897                                                              & 18.81                                                                                                                                            \\
        Reranking (k=5) & 0.9892                                                              & 19.22                                                                                                                                            \\ \hline
    \end{tabular}
    \caption{Ablation Study kết quả reranking mô hình kết hợp trên tập dữ liệu Pittsburgh250k-test \cite{6618963}}
    \label{tab:reranking}
\end{table}
\egroup

% Ablation study Weighted Pose
\subsection{Trung bình trọng số trên mô hình kết hợp}

Chúng tôi áp dụng trung bình trọng số trên tập dữ liệu Pittsburgh250k-test \cite{6618963} và đánh giá kết quả tại bảng \ref{tab:weighted_pose}. Tương tự với các vấn đề đã được đề cập đến trong hướng tiếp cận reranking, với một tập dữ liệu đa dạng về góc nhìn tại một vị trí, việc sử dụng danh sách ảnh truy xuất chứa ảnh truy xuất chính xác và ảnh không chính xác dẫn đến sai lệch đáng kể kết quả tính toán của mô hình. Điều này được thể hiện rõ thông qua xu hướng gia tăng của sai số vị trí khi tăng số ảnh được truy xuất và sử dụng trong tính toán. Vấn đề này nhấn mạnh độ quan trọng trong việc đánh giá và chọn lọc danh sách ảnh truy xuất, loại bỏ ảnh không chính xác.

% table
\bgroup
\def\arraystretch{1.4}%
\setlength\tabcolsep{10 pt}
\begin{table}[H]
    \centering
    \begin{tabular}{|l|c|c|c|}
        \hline
                              & \begin{tabular}[c]{@{}c@{}}Success\\ Rate $\uparrow$\end{tabular} & \begin{tabular}[c]{@{}c@{}}Position\\ (meters) $\downarrow$\end{tabular} \\ \hline
        Weighted Pose (k = 1) & 0.9886                                                              & 18.32                                                                                                                                            \\
        Weighted Pose (k = 3) & 0.9978                                                              & 22.15                                                                                                                                            \\
        Weighted Pose (k = 5) & 0.9996                                                              & 25.48                                                                                                                                            \\ \hline
    \end{tabular}
    \caption{Ablation Study kết quả mô hình kết hợp áp dụng trung bình trọng số trên tập dữ liệu Pittsburgh250k-test \cite{6618963}}
    \label{tab:weighted_pose}
\end{table}
\egroup

% Ablation study Best Overlap
\subsection{Tinh chỉnh mô hình theo hướng độ trùng lặp cao}

Chúng tôi thử nghiệm việc tinh chỉnh lại mô hình VPR theo hướng ưu tiên các ảnh có độ trùng lặp cao và đánh giá kết quả ở bảng \ref{tab:best_overlap}. Qua đó, có thể thấy, hướng tiếp cận này không thể hiện tiềm năng phát triển khả quan. Vì module VPR vốn đã có thể chọn ảnh tốt nhất cả về vị trí lẫn độ trùng lặp, chúng tôi dự đoán việc ảnh có độ trùng lặp cao được chọn nhưng tác vụ RPR lại không thể tính toán tư thế chính xác có thể được là kết quả của quá trình tích lũy lỗi gây nên bởi sai số đến từ các mô hình cơ sở trong tác vụ RPR.

\bgroup
\def\arraystretch{1.4}%
\setlength\tabcolsep{10 pt}
\begin{table}[H]
    \centering
    \begin{tabular}{|l|c|c|c|}
        \hline
                              & \begin{tabular}[c]{@{}c@{}}Success\\ Rate $\uparrow$\end{tabular} & \begin{tabular}[c]{@{}c@{}}Position\\ (meters) $\downarrow$\end{tabular}  \\ \hline
        Base Training         & 0.9886                                                              & 18.32                                                                                                                                       \\
        Best Overlap Training & 0.9866                                                              & 26.92                                                                                                                                         \\ \hline
    \end{tabular}
    \caption{Ablation Study kết quả mô hình kết hợp khi áp dụng huấn luyện ưu tiên trùng lặp trên tập dữ liệu Pittsburgh250k-test \cite{6618963}}
    \label{tab:best_overlap}
\end{table}
\egroup

\section{Kết luận}

Chúng tôi đã đo đạc và so sánh phương pháp đề xuất của nhóm với các mô hình khác nhau. Mô hình đề xuất kết hợp được độ tổng quát hóa cao của tác vụ nhận diện địa điểm trực quan và độ chính xác của tác vụ hồi quy tư thế, đạt kết quả cạnh tranh trong tác vụ định vị trực quan trên dữ liệu unseen. Việc sử dụng mô hình cơ sở giúp chúng tôi giảm thiểu tài nguyên cũng như thời gian huấn luyện mô hình. Đồng thời, việc sử dụng các mô hình cơ sở cho tác vụ RPR giúp mô hình hoạt động tốt hơn với nhiều tập dữ liệu khác nhau so với các mô hình học sâu. Thêm vào đó, việc áp dụng các ngưỡng tương quan và chiến lược khai phá mới đã giúp chúng tôi tiếp tục cải thiện hiệu quả của mô hình.

Qua quá trình thí nghiệm, đo đạc và đánh giá, chúng tôi cũng phát hiện được những điểm cần cải thiện của mô hình. Việc sử dụng các mô hình cơ sở được huấn luyện sẵn tiềm ẩn nhiều rủi ro lỗi tích lũy qua các bước của mô hình, dẫn đến sai số lớn ở kết quả đầu ra. Thêm vào đó, thời gian thực thi của nhiều mô hình cơ sở làm tăng thời gian thực thi của toàn thể mô hình kết hợp. Ngoài ra, các mô hình feature matching thường hoạt động tốt trong môi trường trong nhà, hoặc với tập ảnh có một vật thể cố định làm cột mốc và gặp khó khăn khi được áp dụng cho ảnh thành thị. Có thể thấy, các hướng tiếp cận, trên lý thuyết có tiềm năng cải thiện mô hình, phần nhiều vẫn chưa thể hiện ảnh hưởng tích cực đến độ hiệu quả chung của mô hình trên dữ liệu thành thị phạm vi rộng và thưa.
